
\documentclass[a4paper]{article}

% algumas packages para arrumar as tables co a margin:
% allows for temporary adjustment of side margins
\usepackage{changepage}

% provides filler text
\usepackage{lipsum}

% just makes the table prettier (see \toprule, \bottomrule, etc. commands below)
\usepackage{booktabs}

\usepackage[english, brazil]{babel} %Para traduzir os textos
\usepackage[utf8]{inputenc} %Para poder usar acentos
\usepackage[a4paper]{geometry} %Para ajustar a parte geometrica da folha
\geometry{verbose,tmargin=2.5cm,bmargin=2.5cm,lmargin=3cm,rmargin=3cm} % margem cute
\setlength{\parindent}{10mm}
\usepackage{wrapfig} %Biblioteca Matematica/Grafica
\usepackage{mathptmx} %Biblioteca Matematica/Grafica
\renewcommand{\ttdefault}{mathptmx} %Biblioteca Matematica/Grafica
\usepackage{amsmath} %Biblioteca Matematica/Grafica
\usepackage{amssymb} %Biblioteca Matematica/Grafica
\usepackage[11pt]{moresize}% different letters sizes
\usepackage{float}% enables accurate location of tables
\usepackage{caption}% to make personalized captions
\usepackage{graphicx} %Para inclusão de imagens
\usepackage{amsfonts}
\usepackage[T1]{fontenc}

%Packages do Lyx:
	\usepackage{units}

	%% Because html converters don't know tabularnewline
	\providecommand{\tabularnewline}{\\}
%Fim dos Packages do Lyx

\makeatletter
\providecommand{\tabularnewline}{\\} %define

\title{Experimento 6:  Parte B \\ Calorimetria} % main title
\author{F 229 \\ \textsc{Grupo 1}}
\date{26 de Novembro, 2014}

\makeatother

\begin{document}
	% actually starts the document here
	 \maketitle

	% members of the group


	\begin{center}
		\begin{tabular}{rr}
		          \multicolumn{2}{c}{Integrantes}\tabularnewline
		                                  & \tabularnewline
		        Henrique Noronha Facioli  & RA: 157986 \tabularnewline
		        Guilherme Lucas da Silva  & RA: 155618 \tabularnewline
		          Beatriz Sechin Zazulla  & RA: 154779 \tabularnewline
		              Lucas Alves Racoci  & RA: 156331 \tabularnewline
		Isadora Sophia Garcia Rodopoulos  & RA :158018 \tabularnewline
		\end{tabular}
	\par\end{center}

	%%Seria legal se alguém conseguir colocar os logos no footer da pagina, ou então no header...


	\begin{figure}[!ht]
		\begin{minipage}[b]{0.45\linewidth}
			\includegraphics[scale=0.25]{logo-unicamp-name-line-blk-blk-0480.jpg}
		\end{minipage}
		\hspace{0.5cm}
		\begin{minipage}[b]{0.45\linewidth}
			\includegraphics[scale=0.25]{logo-ifgw.png}
		\end{minipage}
	\end{figure}


	\newpage{}


\section{Resumo}

	Neste experimento, estudamos o calorímetro, um instrumento usado na	medição de processos que envolvam trocam de calor, ou seja, mudança	de temperatura, estado ou qualquer outro processo que seja alterado pelo calor no sistema.
	Nesta segunda parte, estudamos e determinamos o calor específico de	três metais distintos e o calor latente de fusão do gelo. Dadas equações já conhecidas do ramo da Calorimetria, obtivemos as equações finais que, no contexto do experimento, foram capazes de relacionar essas grandezas. Valendo-nos de tais leis e dos materiais disponíveis (calorímetro, termômetro e água a diferentes temperaturas), fomos capazes de encontrar os valores procurados, determinando diversas temperaturas de interesse (em estados iniciais e no equilíbrio térmico) e levando em consideração os erros inerentes a toda medição e ao uso de equipamentos reais.

\section{Objetivo}

	Esta segunda parte do expermiento tem como objetivo o estudo e a determinação do calor específico de 3 metais ($c_{M1},c_{M2},c_{M3}$) e do calor latente de fusão do gelo ($L_{Gelo}$). Para tal, conta-se com um calorímetro, um termômetro e água a diferentes temperaturas de modo a encontrarmos as grandezas procuradas através de equações conhecidas,	tais como:

		\begin{equation}
			Q=mc\Delta T
			\label{eq:calor transferido em funcao de c}
		\end{equation}


		\begin{equation}
			Q=mL
			\label{eq:calor latente de fusao}
		\end{equation}


	Onde, em \eqref{eq:calor transferido em funcao de c}, $Q$ corresponde ao calor transferido ou recebido por um material, dada uma massa $m$,	uma variação de temperatura $\Delta T$ e seu calor específico, $c$.	Em \eqref{eq:calor latente de fusao}, $Q$ também corresponde ao calor transferido por um material, porém no contexto de uma mudança de fase, em função de uma massa $m$ e um coeficiente de calor de latencia de variação do estado físico, que no nosso caso será a fusão, chamaremos esse coeficiente de $L$.

\section{Procedimentos e coleta de dados}
	
	\subsection{Materiais utilizados}
		Para realizar o experimento necessitamos dos seguintes materiais:
		
		\begin{enumerate}
			\item Balança analítica 
			\item Calorímetro contendo:
				\begin{enumerate}
					\item Copo 
					\item Alça 
					\item Suporte 
					\item Isolamento 
					\item Tampa
				\end{enumerate}
			\item Três metais a serem identificados 
			\item Aquecedor 
			\item Gelo 
			\item Garrafa térmica 
			\item Béquer 
			\item Termômetro de mercúrio 
		\end{enumerate}
		
	\subsection{Procedimento}

		De posse dos materiais, primeiramente nós medimos a massa do copo do calorímetro vazio que será utilizada em todo o experimento para calcular a massa de água no calorímetro.			
		
		Para a obtenção dessa massa de água, fizemos: $m_{a}=m_{c+\varnothing}-m_{c+a}$	cujo erro fica: $\sigma_{m_{a}}=\sigma_{m}\cdot\sqrt{2}$, já que as massas $m_{c+\varnothing}$ e $m_{c+a}$ têm o mesmo erro instrumental, $\sigma_{m}$ .
		
		É importante ressaltar que, durante o experimento, o calor específico da água ($c_a$) foi considerado como 1 $\unitfrac{cal}{g \cdot ^{\circ}C}$ e o calor específico do gelo ($c_g$) como 0,5 $\unitfrac{cal}{g \cdot ^{\circ}C}$

		\subsubsection{Parte 1: Obtenção do calor específico dos metais}
			Para cada um dos metais a serem identificados, medimos a massa do metal, a massa do copo do calorímetro com água, a temperatura dessa água no copo do calorímetro, que em alguns casos teve que ser resfriada com gelo, a temperatura do metal e a temperatura deequilíbrio.
			
			Para medir a temperatura do metal, primeiro colocamos água no béquer e esquentamos esta água até a fervura (a temperatura aqui não é relevante e portanto nem foi medida), à seguir, transferimos essa água do béquer para a garrafa térmica e colocamos o metal junto, esperamos até o que julgamos ser o equilíbrio entre o metal e esta água quente do béquer e medimos essa temperatura.
			
			Para medir a temperatura de equilíbrio, colocamos o metal já aquecido dentro do copo do calorímetro com água fria e colocamos esse conjunto no calorímetro. Esperamos novamente até o que julgamos ser o equilíbrio do sistema e medimos essa temperatura. Os valores obtidos para esses dados podem ser visualizados na tabela a seguir: 

			\begin{table}[!ht]
				\caption{Valores medidos experimentalmente conforme expresso acima para a obtenção do calor especifico dos metais}
				\hspace{-2,3cm}%
				\begin{tabular}{|c|c|rl|rl|rl|}
					\hline 
					Dado Medido  & %
					\begin{tabular}{c}
						Símbolo \tabularnewline
						Associado\tabularnewline
					\end{tabular} & \multicolumn{2}{c|}{Para o metal $a$} & \multicolumn{2}{c|}{Para o metal $b$} & \multicolumn{2}{c|}{Para o metal $c$}\tabularnewline
					\hline 
					Massa do copo vazio  & $m_{c+\varnothing}$  & $\left(5,970\pm0,001\right)$  & $\cdot10^{1}\unit{g}$  & $\left(5,970\pm0,001\right)$  & $\cdot10^{1}\unit{g}$  & $\left(5,970\pm0,001\right)$  & $\cdot10^{1}\unit{g}$\tabularnewline
					\hline 
					Massa do copo com água  & $m_{c+a}$  & $\left(2,4420\pm0,0001\right)$  & $\cdot10^{2}\unit{g}$  & $\left(2,1320\pm0,0001\right)$  & $\cdot10^{2}\unit{g}$  & $\left(2,1260\pm0,0001\right)$  & $\cdot10^{2}\unit{g}$\tabularnewline
					\hline 
					Massa de água  & $m_{a}$  & $\left(1,8250\pm0,0001\right)$  & $\cdot10^{2}\unit{g}$  & $\left(1,5350\pm0,0001\right)$  & $\cdot10^{2}\unit{g}$  & $\left(1,529\pm0,0001\right)$  & $\cdot10^{2}\unit{g}$\tabularnewline
					\hline 
					Temperatura da água  & $T_{a}$  & $\left(2,00\pm0,05\right)$  & $\cdot10^{1}\unit{^{\circ}C}$  & $\left(1,30\pm0,05\right)$  & $\cdot10^{1}\unit{^{\circ}C}$  & $\left(2,10\pm0,05\right)$  & $\cdot10^{1}\unit{^{\circ}C}$\tabularnewline
					\hline 
					Temperatura do metal  & $T_{q}$  & $\left(8,00\pm0,05\right)$  & $\cdot10^{1}\unit{^{\circ}C}$  & $\left(8,00\pm0,05\right)$  & $\cdot10^{1}\unit{^{\circ}C}$  & $\left(8,20\pm0,05\right)$  & $\cdot10^{1}\unit{^{\circ}C}$\tabularnewline
					\hline 
					Massa do metal  & $m_{M}$  & $\left(9,090\pm0,001\right)$  & $\cdot10^{1}\unit{g}$  & $\left(4,200\pm0,001\right)$  & $\cdot10^{1}\unit{g}$  & $\left(1,0120\pm0,0001\right)$  & $\cdot10^{2}\unit{g}$\tabularnewline
					\hline 
					Temperatura de equilíbrio  & $T_{e}$  & $\left(2,30\pm0,05\right)$  & $\cdot10^{1}\unit{\unit{^{\circ}C}}$  & $\left(2,00\pm0,05\right)$  & $\cdot10^{1}\unit{\unit{^{\circ}C}}$  & $\left(2,30\pm0,05\right)$  & $\cdot10^{1}\unit{\unit{^{\circ}C}}$\tabularnewline
					\hline 
				\end{tabular}
			\end{table}

		\subsubsection{Parte 2: Calor latente de fusão do gelo}
			Para realizar o cálculo que aproximo o calor latente de fusão do gelo, foi necessário, inicialmente, medir a massa da água e sua temperatura inicial e a massa de gelo e sua temperatura inicial.
			
			Em seguida, foi elaborado um sistema no calorímetro com a água e o gelo, que se finalizou com o gelo derretendo completamente - sendo medido, nesse instante, a sua temperatura de equilíbrio.
			
			Dipostos desses procedimentos, concluiu-se os dados suficientes para aproximar o calor latente de fusão do gelo. Os valores obtidos para esses dados podem ser visualizados na tabela a seguir:
			
			\newpage{}
			
			\begin{table}[!ht]
				\caption{Valores medidos experimentalmente conforme expresso acima para a obtenção do calor latente de fusão do gelo}


				\centering{}%
				\begin{tabular}{|c|c|rl|}
					\hline 
					Dado Medido  & %
					\begin{tabular}{c}
						Símbolo \tabularnewline
						Associado\tabularnewline
					\end{tabular} & \multicolumn{2}{c|}{Resultados} \tabularnewline
					\hline 
					Massa do copo vazio  & $m_{c+\varnothing}$  & $\left(5,970\pm0,001\right)$ & $\cdot10^{1}\unit{g}$   \tabularnewline
					\hline 
					Massa do copo com água  & $m_{c+a}$  & $\left(1,9080\pm0,0001\right)$ & $\cdot10^{2}\unit{g}$   \tabularnewline
					\hline 
					Massa de água  & $m_{a}$  & $\left(1,3110\pm0,0001\right)$ & $\cdot10^{2}\unit{g}$ \tabularnewline
					\hline 
					Temperatura da água  & $T_{a}$  & $\left(3,90\pm0,05\right)$ & $\cdot10^{1}\unit{^{\circ}C}$ \tabularnewline
					\hline 
					Massa do béquer vazio  & $m_{b+\varnothing}$  & $\left(5,82\pm0,01\right)$ & $\cdot10^{1}\unit{g}$ \tabularnewline
					\hline 
					Massa do béquer com gelo  & $m_{b+a}$  & $\left(1,245\pm0,001\right)$ & $\cdot10^{2}\unit{g}$ \tabularnewline
					\hline 
					Massa de gelo & $m_{g}$  & $\left(6,63\pm0,01\right)$ & $\cdot10^{1}\unit{g}$ \tabularnewline
					\hline 
					Temperatura do gelo  & $T_{g}$  & $\left(-3,00\pm0,5\right)$ & $\cdot10^{0}\unit{^{\circ}C}$ \tabularnewline
					\hline 
					Temperatura de equilíbrio  & $T_{e}$  & $\left(2,30\pm0,05\right)$ & $\cdot10^{1}\unit{^{\circ}C}$ \tabularnewline
					\hline 
				\end{tabular}
			\end{table}


	\section{Análise dos resultados}

		\subsection{Cálculo do calor específico dos metais}

			\subsubsection{Encontrando o calor específico de cada metal}

				Evoluindo da fórmula: $\sum Q=0$ - já que o sistema é quase fechado, têm-se, para cada metal:

				\begin{equation}
				Q_{calor\acute{\imath}metro}+Q_{\acute{a}gua}+Q_{metal}=0
				\end{equation}

				\[
				C\cdot\left(T_{e}-T_{a}\right)+c_{a}m_{a}\left(T_{e}-T_{a}\right)+c_{M}m_{M}\left(T_{e}-T_{q}\right)=0
				\]


				\[
				\left(T_{e}-T_{a}\right)\cdot\left(C+c_{a}m_{a}\right)=c_{M}m_{M}\left(T_{q}-T_{e}\right)
				\]


				\begin{equation}
				c_{M}=\cfrac{C+c_{a}m_{a}}{m_{M}}\cdot\cfrac{T_{e}-T_{a}}{T_{q}-T_{e}}\label{eq:c_m}
				\end{equation}


				Assim, fazendo $\Delta T_{e\rightarrow q}=T_{q}-T_{e}$ e $\Delta T_{a\rightarrow e}=T_{e}-T_{a}$
				o erro da equação \eqref{eq:c_m} é dado por:
                
                \begin{equation}
                \hspace{-1,5cm}%
				\sigma_{c_{M}}^{2}=\left(\sigma_{C}\cdot\cfrac{\partial c_{M}}{\partial C}\right)^{2}+\left(\sigma_{m_{a}}\cdot\cfrac{\partial c_{M}}{\partial m_{a}}\right)^{2}+\left(\sigma_{m_{M}}\cdot\cfrac{\partial c_{M}}{\partial m_{M}}\right)^{2}+\left(\sigma_{\Delta T_{e\rightarrow M}}\cdot\cfrac{\partial c_{M}}{\partial\left(\Delta T_{e\rightarrow q}\right)}\right)^{2}+\left(\sigma_{\Delta T_{a\rightarrow e}}\cdot\cfrac{\partial c_{M}}{\partial\left(\Delta T_{a\rightarrow e}\right)}\right)^{2}
				\end{equation}


				\[
				\sigma_{c_{M}}^{2}=\left(\cfrac{T_{e}-T_{a}}{T_{q}-T_{e}}\cdot\cfrac{\sigma_{C}}{m_{M}}\right)^{2}+\left(\cfrac{T_{e}-T_{a}}{T_{q}-T_{e}}\cdot\cfrac{\sigma_{m_{a}}\cdot c_{a}}{m_{M}}\right)^{2}+\left(c_{M}\cdot\cfrac{\sigma_{m_{M}}}{m_{M}}\right)^{2}+\left(c_{M}\cdot\cfrac{\sigma_{\Delta T_{e\rightarrow M}}}{\Delta T_{e\rightarrow q}}\right)^{2}+\left(c_{M}\cdot\cfrac{\sigma_{\Delta T_{a\rightarrow e}}}{\Delta T_{a\rightarrow e}}\right)^{2}
				\]


				\begin{equation}
				\sigma_{c_{M}}=\left(\cfrac{T_{e}-T_{a}}{T_{q}-T_{e}}\right)\cdot\cfrac{\sqrt{\sigma_{C}^{2}+\left(\sigma_{m_{a}}c_{a}\right)^{2}}}{m_{M}^{2}}+c_{m}\cdot\sqrt{\cfrac{\sigma_{m_{M}}^{2}}{m_{M}^{2}}+\cfrac{\sigma_{\Delta T}^{2}}{\left(T_{q}-T_{e}\right)^{2}}+\cfrac{\sigma_{\Delta T}^{2}}{\left(T_{e}-T_{a}\right)^{2}}}
				\end{equation}


				Lembrando da parte A desse experimento, têm-se que:

				\begin{equation}
				C=\left(1,8\pm0,8\right)\unit{\cdot10\cdot\frac{cal}{K}=\left(1,8\pm0,8\right)\cdot10\cdot}\frac{cal}{^{\circ}C}
				\end{equation}


				Assim obtém-se:

				\begin{table}[!ht]
				\caption{Valores calculados de $c_{M}\pm\sigma_{c_{M}}$ para cada metal}


				\centering{}%
				\begin{tabular}{|c|rl|}
				\cline{2-3} 
				\multicolumn{1}{c|}{} & \multicolumn{2}{c|}{$c_{M}\pm\sigma_{c_{M}}$}\tabularnewline
				\hline 
				Para o metal $a$  & $\left(1,2\pm0,3\right)$  & $\cdot10^{-1}\unitfrac{cal}{g\cdot^{\circ}C}$\tabularnewline
				\hline 
				Para o metal $b$  & $\left(4,8\pm0,5\right)$  & $\cdot10^{-1}\unitfrac{cal}{g\cdot^{\circ}C}$\tabularnewline
				\hline 
				Para o metal $c$  & $\left(6\pm2\right)$  & $\cdot10^{-2}\unitfrac{cal}{g\cdot^{\circ}C}$\tabularnewline
				\hline 
				\end{tabular}
				\end{table}

			\subsubsection{Análise crítica dos resultados}

				Não nos foi fornecido o nome dos metais, por isso não é possível estabelecer
				com certeza se o experimento foi bem sucedido. Porém, o valor encontrado
				para o metal $a$ bate com o valor teórico esperado para 
				\begin{itemize}
				\item o ferro ($1,1\cdot10^{-1}\unitfrac{cal}{g\cdot^{\circ}C}$) e 
				\item o cobre($9,4\cdot10^{-2}\unitfrac{cal}{g\cdot^{\circ}C}$). 
				\end{itemize}
				O valor obtido para o metal $c$ também bate com o valor teórico esperado
				para a Prata ($5,6\cdot10^{-2}\unitfrac{cal}{g\cdot^{\circ}C}$).

				Porém não foi possível associar o valor encontrado para o metal $b$
				com o valor teórico esperado para nehum metal que se parecesse visualmente
				com ele. 


				\subparagraph{Justificativa para o erro:}

				O valor do segundo metal não estar de acordo com o esperado, pode
				ser explicado por:
				\begin{enumerate}
					\item Quanto aos materiais:
					\begin{enumerate}
						\item A imperfeição do calorímetro para trocar calor pode
						não ter sido capturada totalmente pelo erro na capacidade 
						térmica; 
						\item Provavelmente o calorímetro usado na parte B do 
						experimento não é exatamente o mesmo do usado na parte A, 
						mas ainda que seja o mesmo calorímetro, o simples manuseio 
						de outros grupos pode ter mudado sua capcidade térmica;
						\item Um dos motivos para a suspeita de que o calorímetro 
						não seja o mesmo, é porque notamos maior dificuldade para 
						fechar este caloŕimetro em comparação com o usado na parte A.
					\end{enumerate}
					\item Quanto ao erro humano:
					\begin{enumerate}
						\item É possível que o metal não tenha ficado tempo 
						suficiente na água quente para que o equilíbrio fosse 
						estabelecido e portanto a temperatura usada nos cálculos 
						para o metal quente pode ter sido superestimada;
						\item Em alguns casos a temperatura da água fria foi 
						medida muito antes de o metal ser introduzido no sistema, 
						então esta água pode ter uma temperatura real no início do 
						processo superior a medida, assim a temperatura usada nos cálculos talvez esteja subestimada.
					\end{enumerate}
					\item Quanto ao processo experimental:
					\begin{enumerate}
						\item O processo de transferência do metal da garrafa 
						térmica para o calorímetro pode tê-lo resfriado e, 
						portanto, a temperatura medida da água quente pode ser 
						relevantemente maior que real do metal, o que também comprometeria os cálculos;
						\item
					\end{enumerate}
					\item Quanto ao experimento anterior, já que a capacidade térmica do calorímetro foi obtida lá:
					\begin{enumerate}
						\item A realização de uma única medida para a capacidade térmica, leva em conta apenas a propagação dos erros instrumentais do termometro e da balança, mas não leva em conta o erro estatístico que provavelmente se manifestaria em decorrência de erros aleatórios como a dissipação de calor por:
						\begin{enumerate}
							\item Irradiação, já que o calorímetro não têm uma superfície suficientemente refletora;
							\item Condução, já que o calorímetro não tem isolamento perfeito.
						\end{enumerate}
						\item Também não foi considerado o erro associado ao calor específico da água.
						\item Ao esquentar a água quente usada para calcular a capacidade térmica parte dela evapora, o que diminui a massa, portanto a medida obtida para a massa também fica superestimada.

					\end{enumerate}
				\end{enumerate}

		\subsection{Calor latente de fusão do gelo}
			Sabendo que:
			\begin{equation}
				Q_{Calor\acute{i}metro} + Q_{\acute{A}gua} + Q_{Gelo} + Q_{Fus\tilde{a}o} + Q_{Gelo \rightarrow \acute{A}gua} = 0
			\end{equation}
			
			...ou seja:
			\hspace{-2cm}%
			$$ C_{c}\Delta T + m_{\acute{A}gua}c_{\acute{A}gua}\Delta T + m_{Gelo}c_{Gelo}\Delta T +  m_{Gelo}L_{f} + m_{Gelo}c_{\acute{A}gua}\Delta T = 0 \leftrightarrow$$
			
			\begin{equation}
			 	L_{f} = -\cfrac{(T_{e}-T_{a})(C_{c}+m_{a}c_{a}) + m_{g}c_{a}T_{e} + m_{g}c_{g}(-T_{g})}{m_g}
			\end{equation}
			  
			com erro dado por erro dado por:
			  
			$\sigma_{L_{f}}^{2} = (\cfrac{\partial L_{f}}{\partial m_a} \cdot \sigma m_{a})^2 + (\cfrac{\partial L_{f}}{\partial m_g} \cdot \sigma m_{g})^2 + (\cfrac{\partial L_{f}}{\partial \Delta T} \cdot \sigma \Delta T)^2 + (\cfrac{\partial L_{f}}{\partial C_c} \cdot \sigma C_c)^2 + (\cfrac{\partial L_{f}}{\partial c_a} \cdot \sigma c_{a})^2 + (\cfrac{\partial L_{f}}{\partial c_g} \cdot \sigma c_{g})^2$
			  
			mas $\sigma c_{g},\sigma c_{a}, \sigma C_{c}, \sigma \Delta T,$ é 0, então
			
			\begin{equation}  
				\sigma_{L_{f}}^{2} = (\cfrac{\partial L_{f}}{\partial m_a} \cdot \sigma m_{a})^2 + (\cfrac{\partial L_{f}}{\partial m_g} \cdot \sigma m_{g})^2
			\end{equation}
			
			Utilizando os valores obtidos experimentalmente da Tabela 2:
			  
			  $$L_{f}\pm\sigma_{L_{f}} = 11,48190045\pm \frac{cal}{g}$$
			  
			O valor do calor latente da água é de 79.53 $\frac{cal}{g}$, um pouco longe do nosso valor...(hahahaha)
			  
			
	\section{Conclusão}

		Conlusão Here - DEU BOSTA!
\end{document}
