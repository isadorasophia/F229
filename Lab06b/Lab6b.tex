\documentclass[a4paper]{article}

% algumas packages para arrumar as tables co a margin:
% allows for temporary adjustment of side margins
\usepackage{changepage}

% provides filler text
\usepackage{lipsum}

% just makes the table prettier (see \toprule, \bottomrule, etc. commands below)
\usepackage{booktabs}

\usepackage[english, brazil]{babel} %Para traduzir os textos
\usepackage[utf8]{inputenc} %Para poder usar acentos
\usepackage[a4paper]{geometry} %Para ajustar a parte geometrica da folha
\geometry{verbose,tmargin=2.5cm,bmargin=2.5cm,lmargin=3cm,rmargin=3cm} % margem cute
\setlength{\parindent}{10mm}
\usepackage{wrapfig} %Biblioteca Matematica/Grafica
\usepackage{mathptmx} %Biblioteca Matematica/Grafica
\renewcommand{\ttdefault}{mathptmx} %Biblioteca Matematica/Grafica
\usepackage{amsmath} %Biblioteca Matematica/Grafica
\usepackage{amssymb} %Biblioteca Matematica/Grafica
\usepackage[11pt]{moresize}% different letters sizes
\usepackage{float}% enables accurate location of tables
\usepackage{caption}% to make personalized captions
\usepackage{graphicx} %Para inclusão de imagens
\usepackage{amsfonts}
\usepackage[T1]{fontenc}

%Packages do Lyx:
	\usepackage{units}

	%% Because html converters don't know tabularnewline
	\providecommand{\tabularnewline}{\\}
%Fim dos Packages do Lyx

\makeatletter
\providecommand{\tabularnewline}{\\} %define

\title{Experimento 6:  Parte B \\ Calorimetria} % main title
\author{F 229 \\ \textsc{Grupo 1}}
\date{26 de Novembro, 2014}

\makeatother

\begin{document}
	% actually starts the document here
	\maketitle

	% members of the group


	\begin{center}
		\begin{tabular}{rr}
		          \multicolumn{2}{c}{Integrantes}\tabularnewline
		& \tabularnewline
		        Henrique Noronha Facioli  & RA: 157986 \tabularnewline
		        Guilherme Lucas da Silva  & RA: 155618 \tabularnewline
		          Beatriz Sechin Zazulla  & RA: 154779 \tabularnewline
		              Lucas Alves Racoci  & RA: 156331 \tabularnewline
		Isadora Sophia Garcia Rodopoulos  & RA :158018 \tabularnewline
		\end{tabular}
	\par\end{center}

	%%Seria legal se alguém conseguir colocar os logos no footer da pagina, ou então no header...


	\begin{figure}[!ht]
		\begin{centering}
			\includegraphics[clip,scale=0.4]{logo-ifgw} 
		\par\end{centering}

		\vspace*{1cm}

		\centering{}
			\includegraphics[clip,scale=0.3]{logo-unicamp-name-line-blk-blk-0480} 
	\end{figure}


	\newpage{}


	\section{Resumo}

	Neste experimento, estudamos o calorímetro, um instrumento usado na
	medição de processos que envolvam trocam de calor, ou seja, mudança
	de temperatura, estado ou qualquer outro processo que seja alterado
	pelo calor no sistema.

	Nesta segunda parte, estudamos e determinamos o calor específico de
	três metais distintos e o calor latente de fusão do gelo. Dadas equações
	já conhecidas do ramo da Calorimetria, obtivemos as equações finais
	que, no contexto do experimento, foram capazes de relacionar essas
	grandezas. Valendo-nos de tais leis e dos materiais disponíveis (calorímetro,
	termômetro e água a diferentes temperaturas), fomos capazes de encontrar
	os valores procurados, determinando diversas temperaturas de interesse
	(em estados iniciais e no equilíbrio térmico) e levando em consideração
	os erros inerentes a toda medição e ao uso de equipamentos reais.


	\section{Objetivo}

	Esta segunda parte do expermiento tem como objetivo o estudo e a determinação
	do calor específico de 3 metais ($c_{M1},c_{M2},c_{M3}$) e do calor
	latente de fusão do gelo ($L_{Gelo}$). Para tal, conta-se com um
	calorímetro, um termômetro e água a diferentes temperaturas de modo
	a encontrarmos as grandezas procuradas através de equações conhecidas,
	tais como:

	\begin{equation}
	Q=mc\Delta T\label{eq:calor transferido em funcao de c}
	\end{equation}


	\begin{equation}
	Q=mL\label{eq:calor latente de fus=00003D0000E3o}
	\end{equation}


	Onde, em \eqref{eq:calor transferido em funcao de c}, $Q$ corresponde
	ao calor transferido ou recebido por um material, dada uma massa $m$,
	uma variação de temperatura $\Delta T$ e seu calor específico, $c$.

	Em \eqref{eq:calor latente de fus=00003D0000E3o}, $Q$ também corresponde
	ao calor transferido por um material, porém no contexto de uma mudança
	de fase, em função de uma massa $m$ e um coeficiente de calor de
	latencia de variação do estado físico, que no nosso caso será a fusão,
	chamaremos esse coeficiente de $L$.


	\section{Procedimentos e coleta de dados}

		Para realizar o experimento necessitamos dos seguintes materiais:


		\subsection{Materiais utilizados}
			\begin{enumerate}
			\item Balança analítica
			\item Calorímetro contendo: 

			\begin{enumerate}
				\item Copo 
				\item Alça
				\item Suporte
				\item Isolamento
			\end{enumerate}
			\item Três metais a serem identificados 
			\item Aquecedor 
			\item Gelo 
			\item Garrafa térmica 
			\item Béquer
			\item Termômetro de mercúrio 
			\end{enumerate}

		\subsection{Procedimento}

		De posse dos materiais, primeiramente nós medimos a massa do copo
		do calorímetro vazio que será utilizada em todo o experimento para
		calcular a massa de água no calorímetro.


			\subsubsection{Parte 1: Identificação dos Metais}

				Para cada um dos metais a serem identificados, medimos a massa do
				metal, a massa do copo do calorímetro com água, a temperatura da água,
				a temperatura do metal e a temperatura de equilíbrio. 

				Para medir a temperatura do metal, primeiro colocamos água no béquer
				junto com o metal e esquentamos a água até a fervura (a temperatura
				aqui não é relevante e portanto nem foi medida), à seguir, esperamos
				o metal entrar em equilíbrio com esta água quente do béquer e medimos
				essa temperatura. 

				Para medir a temperatura de equilíbrio colocamos o metal juntamente
				com o copo do calorímetro com água fria no calorímetro, esperamos
				o sistema entrar em equilíbro e medimos essa temperatura. Os valores
				obtidos para esses dados podem ser visualizados na tabela a seguir: 

				Para a obtenção da massa de água, fizemos: $m_{a}=m_{c+\varnothing}-m_{c+a}$
				cujo erro fica: $\sigma_{m_{a}}=\sigma_{m}\cdot\sqrt{2}$, já que
				as massas $m_{c+\varnothing}$ e $m_{c+a}$ têm o mesmo erro instrumental.

				\begin{table}[!ht]
					\caption{Valores medidos experimentalmente conforme expresso acima}

					\hspace{-2,25cm}%
					\begin{tabular}{|c|c|rl|rl|rl|}
						\hline 
						Dado Medido & %
						\begin{tabular}{c}
						Símbolo \tabularnewline
						Associado\tabularnewline
						\end{tabular} & \multicolumn{2}{c|}{Para o metal $a$} & \multicolumn{2}{c|}{Para o metal $b$} & \multicolumn{2}{c|}{Para o metal $c$}\tabularnewline
						\hline 
						Massa do copo vazio & $m_{c+\varnothing}$ & $\left(5,970\pm0,001\right)$ & $\cdot10^{1}\unit{g}$ & $\left(5,970\pm0,001\right)$ & $\cdot10^{1}\unit{g}$ & $\left(5,970\pm0,001\right)$ & $\cdot10^{1}\unit{g}$\tabularnewline
						\hline 
						Massa do copo com água & $m_{c+a}$ & $\left(2,4420\pm0,0001\right)$ & $\cdot10^{2}\unit{g}$ & $\left(2,1320\pm0,0001\right)$ & $\cdot10^{2}\unit{g}$ & $\left(2,1260\pm0,0001\right)$ & $\cdot10^{2}\unit{g}$\tabularnewline
						\hline 
						Massa de água & $m_{a}$ & $\left(1,8250\pm0,0001\right)$ & $\cdot10^{2}\unit{g}$ & $\left(1,5350\pm0,0001\right)$ & $\cdot10^{2}\unit{g}$ & $\left(1,529\pm0,0001\right)$ & $\cdot10^{2}\unit{g}$\tabularnewline
						\hline 
						Temperatura da água & $T_{a}$ & $\left(2,00\pm0,05\right)$ & $\cdot10^{1}\unit{^{\circ}C}$ & $\left(1,30\pm0,05\right)$ & $\cdot10^{1}\unit{^{\circ}C}$ & $\left(2,10\pm0,05\right)$ & $\cdot10^{1}\unit{^{\circ}C}$\tabularnewline
						\hline 
						Temperatura do metal & $T_{q}$ & $\left(8,00\pm0,05\right)$ & $\cdot10^{1}\unit{^{\circ}C}$ & $\left(8,00\pm0,05\right)$ & $\cdot10^{1}\unit{^{\circ}C}$ & $\left(8,20\pm0,05\right)$ & $\cdot10^{1}\unit{^{\circ}C}$\tabularnewline
						\hline 
						Massa do metal & $m_{M}$ & $\left(9,090\pm0,001\right)$ & $\cdot10^{1}\unit{g}$ & $\left(4,200\pm0,001\right)$ & $\cdot10^{1}\unit{g}$ & $\left(1,0120\pm0,0001\right)$ & $\cdot10^{2}\unit{g}$\tabularnewline
						\hline 
						Temperatura de Equilíbrio & $T_{e}$ & $\left(2,30\pm0,05\right)$ & $\cdot10^{1}\unit{\unit{^{\circ}C}}$ & $\left(2,00\pm0,05\right)$ & $\cdot10^{1}\unit{\unit{^{\circ}C}}$ & $\left(2,30\pm0,05\right)$ & $\cdot10^{1}\unit{\unit{^{\circ}C}}$\tabularnewline
						\hline 
					\end{tabular}
				\end{table}



			\subsubsection{Parte 2: Calor latente de fusão do gelo}


	\section{Análise dos resultados}


		\subsection{Identificação dos Metais usando calor específico.}

			Evoluindo a fórmula:$\sum Q=0$, já que o sistema é quase fechado,
			tem-se, para cada metal:

			\begin{equation}
				Q_{calor\acute{\imath}metro}+Q_{\acute{a}gua}+Q_{metal}=0
			\end{equation}


			\[
				C\cdot\left(T_{e}-T_{a}\right)+c_{a}m_{a}\left(T_{e}-T_{a}\right)+c_{M}m_{M}\left(T_{e}-T_{q}\right)=0
			\]


			\[
				\left(T_{e}-T_{a}\right)\cdot\left(C+c_{a}m_{a}\right)=c_{M}m_{M}\left(T_{q}-T_{e}\right)
			\]


			\begin{equation}
				c_{M}=\cfrac{C+c_{a}m_{a}}{m_{M}}\cdot\cfrac{T_{e}-T_{a}}{T_{q}-T_{e}}\label{eq:c_m}
			\end{equation}


			Assim, fazendo $\Delta T_{e\rightarrow q}=T_{q}-T_{e}$ e $\Delta T_{a\rightarrow e}=T_{e}-T_{a}$
			o erro da equação \eqref{eq:c_m} é dado por:

		    
		    \begin{equation}
				\hspace{-1,5cm}
				\sigma_{c_{M}}^{2}=\left(\sigma_{C}\cdot\cfrac{\partial c_{M}}{\partial C}\right)^{2}+\left(\sigma_{m_{a}}\cdot\cfrac{\partial c_{M}}{\partial m_{a}}\right)^{2}+\left(\sigma_{m_{M}}\cdot\cfrac{\partial c_{M}}{\partial m_{M}}\right)^{2}+\left(\sigma_{\Delta T_{e\rightarrow M}}\cdot\cfrac{\partial c_{M}}{\partial\left(\Delta T_{e\rightarrow q}\right)}\right)^{2}+\left(\sigma_{\Delta T_{a\rightarrow e}}\cdot\cfrac{\partial c_{M}}{\partial\left(\Delta T_{a\rightarrow e}\right)}\right)^{2}
			\end{equation}


			\[
				\sigma_{c_{M}}^{2}=\left(\cfrac{T_{e}-T_{a}}{T_{q}-T_{e}}\cdot\cfrac{\sigma_{C}}{m_{M}}\right)^{2}+\left(\cfrac{T_{e}-T_{a}}{T_{q}-T_{e}}\cdot\cfrac{\sigma_{m_{a}}\cdot c_{a}}{m_{M}}\right)^{2}+\left(c_{M}\cdot\cfrac{\sigma_{m_{M}}}{m_{M}}\right)^{2}+\left(c_{M}\cdot\cfrac{\sigma_{\Delta T_{e\rightarrow M}}}{\Delta T_{e\rightarrow q}}\right)^{2}+\left(c_{M}\cdot\cfrac{\sigma_{\Delta T_{a\rightarrow e}}}{\Delta T_{a\rightarrow e}}\right)^{2}
			\]


			\begin{equation}
				\sigma_{c_{M}}=\left(\cfrac{T_{e}-T_{a}}{T_{q}-T_{e}}\right)\cdot\cfrac{\sqrt{\sigma_{C}^{2}+\left(\sigma_{m_{a}}c_{a}\right)^{2}}}{m_{M}^{2}}+c_{m}\cdot\sqrt{\cfrac{\sigma_{m_{M}}^{2}}{m_{M}^{2}}+\cfrac{\sigma_{\Delta T}^{2}}{\left(T_{q}-T_{e}\right)^{2}}+\cfrac{\sigma_{\Delta T}^{2}}{\left(T_{e}-T_{a}\right)^{2}}}
			\end{equation}


			Lembrando da parte A desse experimento tem-se que:

			\begin{equation}
				C=1,8\pm0,8\unit{\frac{cal}{K}=1,8\pm0,8}\frac{cal}{^{\circ}C}
			\end{equation}


			Assim obtem-se:

			\begin{table}[!ht]
				\caption{Valores calculados de $c_{M}\pm\sigma_{c_{M}}$ para cada metal}

				\centering{}%
				\begin{tabular}{|c|rl|}
					\cline{2-3} 
					\multicolumn{1}{c|}{} & \multicolumn{2}{c|}{$c_{M}\pm\sigma_{c_{M}}$}\tabularnewline
					\hline 
					Para o metal $a$ & $\left(1,1\pm0,3\right)$ & $\cdot10^{-1}\unitfrac{cal}{g\cdot^{\circ}C}$\tabularnewline
					\hline 
					Para o metal $b$ & $\left(4,3\pm0,4\right)$ & $\cdot10^{-1}\unitfrac{cal}{g\cdot^{\circ}C}$\tabularnewline
					\hline 
					Para o metal $c$ & $\left(5\pm2\right)$ & $\cdot10^{-2}\unitfrac{cal}{g\cdot^{\circ}C}$\tabularnewline
					\hline 
				\end{tabular}
			\end{table}


	\section{Conclusão}

	Conlusão Here 
\end{document}
