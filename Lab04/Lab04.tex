% University/School Laboratory Report
% LaTeX Template
% Version 3.1 (25/3/14)
%
% This template has been downloaded from:
% http://www.LaTeXTemplates.com
%
% Original author:
% Linux and Unix Users Group at Virginia Tech Wiki 
% (https://vtluug.org/wiki/Example_LaTeX_chem_lab_report)
%
% License:
% CC BY-NC-SA 3.0 (http://creativecommons.org/licenses/by-nc-sa/3.0/)
%

%	packages and such!

\documentclass{article}

\usepackage{mathptmx} % math functions!
\usepackage{amsmath} % moar math

\usepackage[11pt]{moresize} % different letters sizes
\usepackage{float} % enables accurate location of tables

\usepackage[utf8]{inputenc} % so you can type accented letters

\usepackage{caption} % to make personalized captions

\usepackage{graphicx} % required for the inclusion of images

\setlength{\parindent}{1cm} % paragraph indenting

\usepackage[margin=1.45in]{geometry} % margin correction here


\title{Máquina de Atwood \\ Experimento 4} % main title
\author{F 229 \\ \textsc{Grupo 1}}
\date{XX de XX, 2014}


\begin{document} % actually starts the document here
\maketitle

% members of the group
\begin{center}
	\begin{tabular}{l r l}
		Integrantes: & Henrique Noronha Facioli & 157986 \\
		& Guilherme Lucas da Silva & 155618 \\
		& Beatriz Sechin Zazulla & 154779 \\
		& Lucas Alves Racoci & 156331 \\
		& Isadora Sophia Garcia Rodopoulos & 158018 \\
	\end{tabular}
\end{center}


\section{Resumo}
Neste experimento, estudamos uma \emph{Máquina de Atwood}, um sistema físico que consiste de: um cilindro de latão funcionando como polia, ou seja, com liberdade de girar em torno de um eixo fixo; um fio que será considerado leve - ou seja, com massa irrelevante -, inestensível - isto é, inelástico; dois corpos (1 e 2) que são pendurados na polia por meio do fio anteriormente citado, onde:
\begin{itemize} 
	\item O corpo 1 consiste de um sub-corpo de massa ${m}_{1}$ e mais $n_{1}$ de $5$ sub-corpos; 
	\item O corpo 2 consiste de um sub-corpo de massa ${m}_{2}$ e mais $n_{2}$ de $5$ sub-corpos; 
	\item Os valores de $n_{1}$ e $n_{2}$ são tais que $n_{1}+n_{2}=5$; 
	\item As massas dos corpos 1 e 2 serão chamadas respectivamente de $m_{1}$ e $m_{2}$.
\end {itemize} 

Sabemos que a diferença entre as massas dos dois corpos gera um torque não nulo na polia, o que nos permite estudar seu Momento de Inércia $I_{0}$ e a aceleração da grávidade $g$, através da fórmula a seguir:

\[
\Delta m=\cfrac{2h}{gR^{2}}(I+MR^{2})\ensuremath{\cfrac{1}{t^{2}}}+\cfrac{\tau_{a}}{gR}
\]


\section{Objetivo}
Este experimento tem como principal objetivo o estudo da máquina de Atwood através da determinação do momento de inércia da polia e do torque da força de atrito, possibilitados a partir da manipulação de um sistema inserido no modelo de estudo.


\section{Procedimentos e coleta de dados}

Na realização deste experimento foram utilizados os seguintes materiais: 
\begin{enumerate} 
	\item Polia de latão com eixo;
	\item Barbante;
	\item Conjunto de discos metálicos;
	\item Trena;
	\item Paquímetro;
	\item Balança de precisão;
	\item Cronômetro.
 \end {enumerate} 

\begin{table}[!ht]
	\begin{center}
		\caption*{\textbf{Tabela 1:} Modelo de tabela}
		\begin{tabular}{| l | l |}
			\cline{2-2} \multicolumn{0}{c|}{ } & \multicolumn{1}{c|}{\textbf{Massa ($Kg$)}} \\  \cline{1-2}
			\multicolumn{0}{|c|}{\textbf{Medida}} & 1,2790\\ \hline
			\multicolumn{0}{|c|}{\textbf{Erro Instrumental}} & 0,0001\\ \hline
		\end{tabular}
	\end{center}
\end{table}

\section{Análise dos resultados}
Para determinar o momento de inercia $I$ e o torque da força de atrito
$\tau_{a}$ através da equação (1), precisamos escolher quem será
$X$ e $Y$ e depois, se necessário linearizar a fórmula, mas nesse
caso, se fizermos: $\underset{Y}{\underbrace{\Delta m}}=\underset{a}{\underbrace{\cfrac{2h}{gR^{2}}(I+MR^{2})}}\underset{X}{\underbrace{\ensuremath{\cfrac{1}{t^{2}}}}}+\underset{b}{\underbrace{\cfrac{\tau_{a}}{gR}}}$
a equção já fica em sua forma linearizada. Assim, variando $\Delta m$
e $t$ obtemos os valores de $I$ e $\tau_{a}$ a partir de $a$,
$b$, $h$, $g$, $R$, e $M=m_{1}+m_{2}$.

Mas para obter $a$ e $b$ precisa-se realizar o Método dos Mínimos
Quadrados, e para isso precisamos achar um valor de $t$ para cada
valor de $\Delta m$.

Calculando primeiro cada um dos 5 $\Delta m$'s tem-se:

\begin{tabular}{|c|c|c|c|c|c|}
\hline 
N & 1 & 2 & 3 & 4 & 5\tabularnewline
\hline 
$m_{1}$$[Kg]$ & $\widetilde{m_{1}}+a+b+c+d+e$ & $\widetilde{m_{1}}+a+b+c+d$ & $\widetilde{m_{1}}+a+b+c$ & $\widetilde{m_{1}}+a+b+c+e$ & $\widetilde{m_{1}}+a+c+e$\tabularnewline
\hline 
$m_{1}$$[Kg]$ & $(9,611\pm0,002)\cdot10^{-1}$ & $(9,513\pm0,002)\cdot10^{-1}$ & $(9,420\pm0,002)\cdot10^{-1}$ & $(9,518\pm0,002)\cdot10^{-1}$ & $(9,321\pm0,002)\cdot10^{-1}$\tabularnewline
\hline 
$m_{2}$$[Kg]$  & $\widetilde{m_{2}}$ & $\widetilde{m_{2}}+e$ & $\widetilde{m_{2}}+d+e$$(\pm)\cdot10^{-1}$ & $\widetilde{m_{2}}+d$ & $\widetilde{m_{2}}+b+d$\tabularnewline
\hline 
$m_{2}$$[Kg]$ & $(8,934\pm0,001)\cdot10^{-1}$ & $(9,032\pm0,001)\cdot10^{-1}$ & $(9,125\pm0,001)\cdot10^{-1}$ & $(9,027\pm0,001)\cdot10^{-1}$ & $(9,224\pm0,002)\cdot10^{-1}$\tabularnewline
\hline 
$\Delta m$ & $(6,77\pm0,03)\cdot10^{-2}$ & $(4,81\pm0,03)\cdot10^{-2}$ & $(2,95\pm0,03)\cdot10^{-2}$ & $(4,91\pm0,03)\cdot10^{-2}$ & $(9,7\pm0,3)\cdot10^{-3}$\tabularnewline
\hline 
\end{tabular}

Como fizemos 5 medidas de $t$ pra cada $\Delta m$, então para achar
o valor único pra $t$ e seu erro devemos fazer:

\[
t=\overline{t}\pm\sigma_{t}
\]
 

onde:

$\overline{t}=\cfrac[r]{\overset{{\scriptstyle 5}}{\underset{{\scriptstyle i=1}}{\sum}}t_{i}}{5}$
é a média aritmética e 

$\sigma_{t}=\sqrt{{\displaystyle \sigma_{t_{inst}}^{2}+\sigma_{t_{estat}}^{2}}}=\sqrt{{\displaystyle \sigma_{t_{inst}}^{2}+}\cfrac[r]{1}{5}\cfrac[l]{1}{4}\overset{{\scriptstyle 5}}{\underset{{\scriptstyle i=1}}{\sum}}(t_{i}+\overline{t})^{2}}$
é o erro total.

Ou seja:

\begin{tabular}{|c|c|c|c|c|c|}
\cline{2-6} 
\multicolumn{1}{c|}{} & $\Delta m_{1}$ & $\Delta m_{2}$ & $\Delta m_{3}$ & $\Delta m_{4}$ & $\Delta m_{5}$\tabularnewline
\hline 
$t_{1}\pm\sigma_{t_{inst}}$$[s]$ &  &  &  &  & \tabularnewline
\hline 
$t_{2}\pm\sigma_{t_{inst}}$$[s]$ &  &  &  &  & \tabularnewline
\hline 
$t_{3}\pm\sigma_{t_{inst}}$$[s]$ &  &  &  &  & \tabularnewline
\hline 
$t_{4}\pm\sigma_{t_{inst}}$$[s]$ &  &  &  &  & \tabularnewline
\hline 
$t_{5}\pm\sigma_{t_{inst}}$$[s]$ &  &  &  &  & \tabularnewline
\hline 
$t=\overline{t}\pm\sigma_{t}$$[s]$ & $2,63\pm0,02$ & $3,22\pm0,03$ & $4,21\pm0,02$ & $3,21\pm0,02$ & $7,58\pm0,08$\tabularnewline
\hline 
\end{tabular}

Fazendo o Método dos Mínimos Quadrados com esses dados obtem-se o
gráfico da reta:

Cujos coeficientes angular e linear são respectivamente:

\begin{eqnarray*}
\begin{cases}
a=\cfrac{2h}{gR^{2}}(I+MR^{2}) & (angular)\\
b=\cfrac{\tau_{a}}{gR} & (linear)
\end{cases} & \Leftrightarrow & \begin{cases}
agR^{2} & =2h(I+MR^{2})\\
bgR & =\tau_{a}
\end{cases}\Leftrightarrow
\end{eqnarray*}


\begin{eqnarray*}
\Leftrightarrow & \begin{cases}
\cfrac[r]{agR^{2}}{2h}=I+MR^{2}\\
\tau_{a}=bgR
\end{cases}\Leftrightarrow & \begin{cases}
I=\cfrac[r]{agR^{2}}{2h}-MR^{2}\\
\tau_{a}=bgR
\end{cases}
\end{eqnarray*}


Para calcular os erros tem-se:

\[
\sigma_{I}^{2}=\sigma_{a}^{2}\left(\cfrac[r]{\partial I}{\partial a}\right)^{2}+\sigma_{g}^{2}\left(\cfrac[r]{\partial I}{\partial g}\right)^{2}+\sigma_{R}^{2}\left(\cfrac{\partial I}{\partial R}\right)^{2}+\sigma_{h}^{2}\left(\cfrac[r]{\partial I}{\partial h}\right)^{2}+\sigma_{M}^{2}\left(\cfrac{\partial I}{\partial M}\right)^{2}
\]


\[
\sigma_{I}=\sqrt{\sigma_{a}^{2}\left(\cfrac[r]{gR^{2}}{2h}\right)^{2}+\sigma_{g}^{2}\left(\cfrac[r]{aR^{2}}{2h}\right)^{2}+\sigma_{R}^{2}\left(\cfrac{Rag}{h}-2RM\right)^{2}+\sigma_{h}^{2}\left(\cfrac[r]{-agR^{2}}{2h^{2}}\right)^{2}+\sigma_{M}^{2}\left(-R^{2}\right)^{2}}
\]


Já para o torque da força de atrito o processo é um pouco mais simples:

\[
\sigma_{\tau_{a}}^{2}=\sigma_{b}^{2}\left(\cfrac[r]{\partial I}{\partial b}\right)^{2}+\sigma_{g}^{2}\left(\cfrac[r]{\partial I}{\partial g}\right)^{2}+\sigma_{h}^{2}\left(\cfrac[r]{\partial I}{\partial h}\right)^{2}
\]


\[
\sigma_{\tau_{a}}=\sqrt{(\sigma_{b}gh)^{2}+(b\sigma_{g}h)^{2}+(bg\sigma_{h})^{2}}
\]


Portanto:

$I=\overline{I}\pm\sigma_{I}=(2,00\pm0,07)\centerdot10^{-1}$$Kg\centerdot m^{2}$

$\tau_{a}=\overline{\tau_{a}}\pm\sigma_{\tau_{a}}=(1,5\pm0,2)\centerdot10^{-2}$$N\centerdot m$

Para termos um parâmetro de comparação podemos usar a massa e o raio
do cilindro de metal para calcular o valor do Momento de Inércia teórico
$I_{T}$e comparar com o que obtivemos experimentalmente:

$I_{T}=M_{polia}\centerdot R^{2}=2,05330\centerdot(4,995\centerdot10^{-1})^{2}=2,5615\centerdot10^{-1}$

Para o erro, como não foi informado o erro para $M_{polia}$supomos
somente o erro instrumental de $10^{-4}$$Kg$ da mesma balança analítica:

\[
\sigma_{I_{T}}^{2}=\sigma_{M_{polia}}^{2}\left(\cfrac{\partial I}{\partial M_{polia}}\right)^{2}+\sigma_{R}^{2}\left(\cfrac{\partial I}{\partial R}\right)^{2}
\]


\[
\sigma_{I_{T}}=\sqrt{\sigma_{M_{polia}}^{2}\left(R^{2}\right)^{2}+\sigma_{R}^{2}\left(2MR\right)^{2}}
\]


Ou seja, temos que:

$I_{T}=(2,6\pm0,5)\centerdot10^{-1}$ $Kg\centerdot m^{2}$

$I=$$(2,00\pm0,07)\centerdot10^{-1}$$Kg\centerdot m^{2}$

Se usarmos todas as casas decimais e não somente as mostradas aqui
temos:

\[
I_{T}-\sigma_{I_{T}}=2,05\centerdot10^{-1}Kg\centerdot m^{2}\leq2,07\centerdot10^{-1}Kg\centerdot m^{2}=I+\sigma_{I}
\]


Isso significa que nosso resultado pode ser considerado dentro do
esperado por esse tipo de parâmetro, mas não pelo parâmetro formalizado
em aula, que não considera as casas decimais de erro não significativas.

Mas alguns fatores podem justificar o valor longe, ainda que muito
perto do esperado, tais como:
\begin{itemize}
	\item O fio não ser inextensível;
	\item É possível que o fio escorregue no cilindro de latão;
\end{itemize}
O que pode ser feito para resolver esses problemas é:
\begin{itemize}
	\item Considerar o fio extensível, o que pode complicar bastante as contas;
	\item Usar uma superfície que tenha um alto coeficiente de atrito estático
	com o fio;
\end{itemize}

\section{Conclusão}

\end{document}
