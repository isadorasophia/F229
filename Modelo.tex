\documentclass[a4paper]{article} %Faz com que a saída padrao seja a4


\usepackage[brazilian]{babel} %Para sair ja escrito, tabela e não table
\usepackage[utf8]{inputenc} % Para usar acentos
\usepackage{mathptmx} % math functions!
\usepackage{amsmath} % moar math
\usepackage[11pt]{moresize} % different letters sizes
\usepackage{float} % enables accurate location of tables
\usepackage{caption} % to make personalized captions
\usepackage{graphicx} % required for the inclusion of images
\usepackage[margin=0.8in]{geometry} %Faz uma correção das margens


\setlength{\parindent}{10mm}


\title{Experimento 0X \\ nome do experimento}
\author{F 229 \\ \textsc{Grupo 1}}
\date{00 de Setembro, 2014}


\begin{document}
\maketitle

\begin{center}
	\begin{tabular}{l r l}
		Integrantes: & Henrique Noronha Facioli & 157986 \\
		& Guilherme Lucas da Silva & 155618 \\
		& Beatriz Sechin Zazulla & 154779 \\
		& Lucas Alves Racoci & 156331 \\
		& Isadora Sophia Garcia Rodopoulos & 158018 \\
	\end{tabular}
\end{center}


\section{Resumo} 

\section{Objetivo}

\section{Procedimentos e coleta de dados}

 \subsection{Materiais utilizados}

\begin{enumerate} 
	\item materiais
\end{enumerate} 
 
 \subsection{Procedimento}
 
\section{Análise e Resultados}

\section{Conclusão}

\end{document}