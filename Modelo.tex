\documentclass[a4paper]{article}

% algumas packages para arrumar as tables co a margin:
% allows for temporary adjustment of side margins
\usepackage{changepage}

% provides filler text
\usepackage{lipsum}

% just makes the table prettier (see \toprule, \bottomrule, etc. commands below)
\usepackage{booktabs}

\usepackage[brazilian]{babel} %Para traduzir os textos
\usepackage[utf8]{inputenc} %Para poder usar acentos
\usepackage[a4paper]{geometry} %Para ajustar a parte geometrica da folha
\geometry{verbose,tmargin=2cm,bmargin=3cm,lmargin=3cm,rmargin=3cm} %Parte de margens
\setlength{\parindent}{0.5cm}
\usepackage{wrapfig} %Biblioteca Matematica/Grafica
\usepackage{mathptmx} %Biblioteca Matematica/Grafica
\renewcommand{\ttdefault}{mathptmx} %Biblioteca Matematica/Grafica
\usepackage{amsmath} %Biblioteca Matematica/Grafica
\usepackage{amssymb} %Biblioteca Matematica/Grafica
\usepackage[11pt]{moresize}% different letters sizes
\usepackage{float}% enables accurate location of tables
\usepackage{caption}% to make personalized captions
\usepackage{graphicx} %Para inclusão de imagens
\usepackage{amsfonts}
\usepackage[T1]{fontenc}

\makeatletter
\providecommand{\tabularnewline}{\\} %define

\title{Máquina de Atwood \\ Experimento 4} % main title
\author{F 229 \\ \textsc{Grupo 1}}
\date{22 de Outubro, 2014}

\begin{document} % actually starts the document here
\maketitle

% members of the group
\begin{center}
	\begin{tabular}{l r l}
		Integrantes:\\\\
		 Henrique Noronha Facioli & RA: 157986 \\
		 Guilherme Lucas da Silva & RA: 155618 \\
		 Beatriz Sechin Zazulla & RA: 154779 \\
		 Lucas Alves Racoci & RA: 156331 \\
		 Isadora Sophia Garcia Rodopoulos & RA :158018 \\
	\end{tabular}
\end{center}

%%Seria legal se alguém conseguir colocar os logos no footer da pagina, ou então no header...

\begin{figure}[!ht]
	\begin{minipage}[b]{0.45\linewidth}
	\includegraphics[scale=0.25]{logo-unicamp-name-line-blk-blk-0480.jpg}
	\end{minipage}
	\hspace{0.5cm}
	\begin{minipage}[b]{0.45\linewidth}
	\includegraphics[scale=0.25]{logo-ifgw.png}
	\end{minipage}
\end{figure}

\newpage

\section{Resumo}

\section{Objetivo}

\section{Procedimentos e coleta de dados}

\subsection{Materiais}

\begin{enumerate}
	\item Items
 \end {enumerate}

\subsection{Procedimento}

\begin{figure}[!ht]
	\centering
	\includegraphics
	\caption{Imagem Experimento}
\end{figure}



\section{Análise dos resultados}

\section{Conclusão}

\end{document}
