\documentclass[english,brazil]{article}
\usepackage{mathptmx}
\renewcommand{\ttdefault}{mathptmx}
\usepackage[T1]{fontenc}
\usepackage[latin9]{inputenc}
\usepackage[a4paper]{geometry}
\geometry{verbose,tmargin=2cm,bmargin=3cm,lmargin=3cm,rmargin=3cm} % margem cute

\setlength{\parindent}{10mm}
\usepackage{float}
\usepackage{units}
\usepackage{amsmath}
\usepackage{amssymb}
\usepackage{graphicx}
\makeatletter
\providecommand{\tabularnewline}{\\}
\newcommand{\lyxdot}{.}
\usepackage{changepage}
\usepackage{lipsum}
\usepackage{booktabs}
\usepackage[english, brazil]{babel}
\renewcommand{\ttdefault}{mathptmx}
\usepackage[11pt]{moresize}
\usepackage{caption}
\usepackage{amsfonts}
\usepackage{units}
\providecommand{\tabularnewline}{\\}


\title{Experimento 5  \\ Viscosidade: Lei de Stokes } % main title
\author{F 229 \\ \textsc{Grupo 1}}
\date{19 de Novembro, 2014}

\makeatother

\usepackage{babel}
\begin{document}
% actually starts the document here
\maketitle

% members of the group


\begin{center}
	\begin{tabular}{rr}
	                    Integrantes:  & \tabularnewline
	                                  & \tabularnewline
	        Henrique Noronha Facioli  & RA: 157986 \tabularnewline
	        Guilherme Lucas da Silva  & RA: 155618 \tabularnewline
	          Beatriz Sechin Zazulla  & RA: 154779 \tabularnewline
	              Lucas Alves Racoci  & RA: 156331 \tabularnewline
	Isadora Sophia Garcia Rodopoulos  & RA :158018 \tabularnewline
	\end{tabular}
\par\end{center}

%%Seria legal se alguém conseguir colocar os logos no footer da pagina, ou então no header...


\begin{figure}[!ht]
	\begin{centering}
		\includegraphics[clip,scale=0.4]{logo-ifgw} 
	\par\end{centering}
	\vspace*{1cm}
	\centering{}
		\includegraphics[clip,scale=0.3]{logo-unicamp-name-line-blk-blk-0480} 
\end{figure}


\newpage{}


\section{Resumo}

	Resumo here

\section{Objetivo}

	Objetivo here

\section{Procedimentos e coleta de dados}


\subsection{Materiais utilizados}
	\begin{enumerate}
		\item Tubo de vidro com glicerina
		\item Suporte com marcas graduadas
		\item Conjunto de esferas
		\item Paquímetro 
		\item Micrômetro
		\item Cronômetro
		\item Termômetro de mercúrio
	\end{enumerate}

\subsection{Procedimento}

	Cinco pequenas esferas de aço são colocadas com a menor velocidade
	inicial conseguida na superfície da mistura de água mais glicerina.

\subsection{Dados brutos:}

\subsubsection{Valores conhecidos}

\begin{table}[H]
%	\caption*{Diâmetros das esferas}
	\centering{}%
	\begin{tabular}{|c|c|}
		\hline 
		Esfera & Diametro das Esferas \tabularnewline
		\hline 
		a & $(2,0\pm0,5)\unit{mm}$\tabularnewline
		\hline 
		b & $(2,5\pm0,5)\unit{mm}$\tabularnewline
		\hline 
		c & $(3,0\pm0,5)\unit{mm}$\tabularnewline
		\hline 
		d & $(3,5\pm0,5)\unit{mm}$\tabularnewline
		\hline 
		e & $(4,0\pm0,5)\unit{mm}$\tabularnewline
		\hline 
	\end{tabular}
\end{table}

\begin{table}[H]
	\caption{Outros Dados}

	\centering{}%
	\begin{tabular}{|r|rl|}
		\hline 
		Dado & Valor & \tabularnewline
		\hline 
		Diametro do Tubo & $(5,325\pm0,005)$ & \selectlanguage{english}%
		$\cdot10^{1}\unit{mm}$\selectlanguage{brazil}%
		\tabularnewline
		\hline 
		Altura & $(3,000\pm0,005)$ & $\cdot10^{1}\unit{cm}$\tabularnewline
		\hline 
		Temperatura & $(2,60\pm0,05)$ & $\cdot10^{1}\unit{^{\circ}C}$\tabularnewline
		\hline 
	\end{tabular}
\end{table}

\subsubsection{Valores obtidos experimentalmente}

	\begin{table}[H]
		\begin{adjustwidth}{-1cm}{-1cm}
		\caption{Tempos de queda de cada esfera pra cada uma das $N=5$ medidas realizadas
		(i.e. $t_{ki}$ é o tempo de queda da k-ésima esfera na i-ésima medição)}

		\centering{}%
		\begin{tabular}{|c|cc|cc|cc|cc|cc|}
			\hline 
			$i$ & $t_{ai}$  &  & $t_{bi}$ &  & \selectlanguage{english}%
			$t_{ci}$\selectlanguage{brazil}%
			 &  & \selectlanguage{english}%
			$t_{di}$\selectlanguage{brazil}%
			 &  & \selectlanguage{english}%
			$t_{ei}$\selectlanguage{brazil}%
			 & \tabularnewline
			\hline 
			$1$ & $(1,713\pm0,001)$ & \selectlanguage{english}%
			$\cdot10^{1}\unit{s}$\selectlanguage{brazil}%
			 & $(1,146\pm0,001)$ & \selectlanguage{english}%
			$\cdot10^{1}\unit{s}$\selectlanguage{brazil}%
			 & $(8,19\pm0,01)$ & \selectlanguage{english}%
			$\cdot\unit{s}$\selectlanguage{brazil}%
			 & $(6,06\pm0,01)$ & \selectlanguage{english}%
			$\cdot\unit{s}$\selectlanguage{brazil}%
			 & $(4,87\pm0,01)$ & \selectlanguage{english}%
			$\cdot\unit{s}$\selectlanguage{brazil}%
			\tabularnewline
			\hline 
			$2$ & $(1,718\pm0,001)$ & \selectlanguage{english}%
			$\cdot10^{1}\unit{s}$\selectlanguage{brazil}%
			 & $(1,150\pm0,001)$ & \selectlanguage{english}%
			$\cdot10^{1}\unit{s}$\selectlanguage{brazil}%
			 & $(8,19\pm0,01)$ & \selectlanguage{english}%
			$\cdot\unit{s}$\selectlanguage{brazil}%
			 & $(6,28\pm0,01)$ & \selectlanguage{english}%
			$\cdot\unit{s}$\selectlanguage{brazil}%
			 & $(4,78\pm0,01)$ & \selectlanguage{english}%
			$\cdot\unit{s}$\selectlanguage{brazil}%
			\tabularnewline
			\hline 
			$3$ & $(1,753\pm0,001)$ & \selectlanguage{english}%
			$\cdot10^{1}\unit{s}$\selectlanguage{brazil}%
			 & $(1,150\pm0,001)$ & \selectlanguage{english}%
			$\cdot10^{1}\unit{s}$\selectlanguage{brazil}%
			 & $(8,21\pm0,01)$ & \selectlanguage{english}%
			$\cdot\unit{s}$\selectlanguage{brazil}%
			 & $(6,06\pm0,01)$ & \selectlanguage{english}%
			$\cdot\unit{s}$\selectlanguage{brazil}%
			 & $(4,75\pm0,01)$ & \selectlanguage{english}%
			$\cdot\unit{s}$\selectlanguage{brazil}%
			\tabularnewline
			\hline 
			$4$ & $(1,734\pm0,001)$ & \selectlanguage{english}%
			$\cdot10^{1}\unit{s}$\selectlanguage{brazil}%
			 & $(1,147\pm0,001)$ & \selectlanguage{english}%
			$\cdot10^{1}\unit{s}$\selectlanguage{brazil}%
			 & $(8,22\pm0,01)$ & \selectlanguage{english}%
			$\cdot\unit{s}$\selectlanguage{brazil}%
			 & $(6,12\pm0,01)$ & \selectlanguage{english}%
			$\cdot\unit{s}$\selectlanguage{brazil}%
			 & $(4,85\pm0,01)$ & \selectlanguage{english}%
			$\cdot\unit{s}$\selectlanguage{brazil}%
			\tabularnewline
			\hline 
			$5$ & $(1,722\pm0,001)$ & \selectlanguage{english}%
			$\cdot10^{1}\unit{s}$\selectlanguage{brazil}%
			 & $(1,138\pm0,001)$ & \selectlanguage{english}%
			$\cdot10^{1}\unit{s}$\selectlanguage{brazil}%
			 & $(8,17\pm0,01)$ & \selectlanguage{english}%
			$\cdot\unit{s}$\selectlanguage{brazil}%
			 & $(6,25\pm0,01)$ & \selectlanguage{english}%
			$\cdot\unit{s}$\selectlanguage{brazil}%
			 & $(4,72\pm0,01)$ & \selectlanguage{english}%
			$\cdot\unit{s}$\selectlanguage{brazil}%
			\tabularnewline
			\hline 
		\end{tabular}
		\end{adjustwidth}
	\end{table}

\section{Análise dos resultados}

	\subsection{Tratamento inicial dos dados}

	\subsubsection{Passando para o Sistema Internacional de Unidades}

		\begin{table}[H]
			\caption{Dados no S.I. com os símbolos associados}


			\centering{}
			\begin{tabular}{|c|c|}
				\hline 
				Raio das esferas ($S\acute{\imath}mbolo$) & Valor \tabularnewline
				\hline 
				$r_{a}$ & $(1,0\pm0,3)\cdot10^{-3}\unit{m}$\tabularnewline
				\hline 
				$r_{b}$ & $(1,3\pm0,3)\cdot10^{-3}\unit{m}$\tabularnewline
				\hline 
				$r_{c}$ & $(1,5\pm0,3)\cdot10^{-3}\unit{m}$\tabularnewline
				\hline 
				$r_{d}$ & $(1,8\pm0,3)\cdot10^{-3}\unit{m}$\tabularnewline
				\hline 
				$r_{e}$ & $(2,0\pm0,3)\cdot10^{-3}\unit{m}$\tabularnewline
				\hline 
			\end{tabular}

		\end{table}

		\begin{table}[H]
				\centering{}
				\begin{tabular}{|r|rl|}
				\hline 
				Dado ($S\acute{\imath}mbolo$) & Valor & \tabularnewline
				\hline 
				Raio do Tubo ($R$) & $(2,663\pm0,003)$ & \selectlanguage{english}%
				$\cdot10^{-2}\unit{m}$\selectlanguage{brazil}%
				\tabularnewline
				\hline 
				Altura ($H$) & $(3,000\pm0,005)$ & $\cdot10^{-1}\unit{m}$\tabularnewline
				\hline 
				Temperatura ($T$) & $(2,60\pm0,05)$ & $\cdot10^{1}\unit{^{\circ}C}$\tabularnewline
				\hline 
			\end{tabular}
		\end{table}

	\subsubsection{Obtendo o tempo médio de queda para cada esfera}

		Para cada esfera k, $k=a,\ldots,e$ tem-se que:

		$t_{k}\pm\sigma_{t_{k}}={\displaystyle \underset{t_{k}}{\underbrace{\left(\sum_{i=1}^{N}t_{ki}\right)}}\pm\underset{\sigma_{t_{k}}}{\underbrace{\sqrt{\underset{\sigma_{t_{k_{instrumental}}}^{2}}{\underbrace{0,01^{2}}}+\underset{\sigma_{t_{k_{estat\acute{\imath}stico}}}^{2}}{\underbrace{\cfrac{1}{N}\cdot\cfrac{1}{N-1}\cdot\sum_{i=1}^{N}\left(t_{k}-t_{ki}\right)^{2}}}}}}}$

		Assim, aplicando essas fórmulas obtem-se:

		\begin{table}[H]
			\caption{Tempo de queda médio de cada esfera}

			\centering{}%
			\begin{tabular}{|c|rl|}
				\hline 
				Tempo de queda médio & Valor  & \tabularnewline
				\hline 
				$t_{a}$ & $(1,728\pm0,007)$ & \selectlanguage{english}%
				$\cdot10^{1}\unit{s}$\selectlanguage{brazil}%
				\tabularnewline
				\hline 
				$t_{b}$ & $(1,146\pm0,002)$ & \selectlanguage{english}%
				$\cdot10^{1}\unit{s}$\selectlanguage{brazil}%
				\tabularnewline
				\hline 
				$t_{c}$ & $(8,20\pm0,01)$ & \selectlanguage{english}%
				$\cdot\unit{s}$\selectlanguage{brazil}%
				\tabularnewline
				\hline 
				$t_{d}$ & $(6,15\pm0,05)$ & \selectlanguage{english}%
				$\cdot\unit{s}$\selectlanguage{brazil}%
				\tabularnewline
				\hline 
				$t_{e}$ & $(4,79\pm0,03)$ & \selectlanguage{english}%
				$\cdot\unit{s}$\selectlanguage{brazil}%
				\tabularnewline
				\hline 
			\end{tabular}
		\end{table}

	\subsection{Cálculo da velocidade $v'_{L}$ através de $t_{k}$e $H$}

		Sabemos que:

		\[
			v'_{L}=\cfrac{H}{t_{k}}
		\]


		e portanto 

		\[
			\sigma_{v'_{L}}=\sqrt{\left(\sigma_{H}\cdot\cfrac{\partial v'_{L}}{\partial H}\right)^{2}+\left(\sigma_{t_{k}}\cdot\cfrac{\partial v'_{L}}{\partial t_{k}}\right)^{2}}
		\]
		\begin{equation}
			=\sqrt{\left(\cfrac{\sigma_{H}}{t_{k}}\right)^{2}+\left(-H\cdot\cfrac{\sigma_{t_{k}}}{t_{k}^{2}}\right)^{2}}=\cfrac{\sqrt{\sigma_{H}^{2}\cdot t_{k}^{2}+H^{2}\cdot\sigma_{t_{k}}^{2}}}{t_{k}^{2}}\label{eq:erro em v'}
		\end{equation}
		assim, usando a fórmula obtemos:

	\begin{table}[H]
		\caption{Velocidade $v'_{L}$calculada atravez das fórmulas acima para cada
		esfera}


		\centering{}%
		\begin{tabular}{|c|rl|}
			\hline 
			Esfera & $v'_{L}$ & \tabularnewline
			\hline 
			a & $(1,736\pm0,008)$ & \selectlanguage{english}%
			$\cdot10^{-2}\unitfrac{m}{s}$\selectlanguage{brazil}%
			\tabularnewline
			\hline 
			b & $(2,617\pm0,007)$ & \selectlanguage{english}%
			$\cdot10^{-2}\unitfrac{m}{s}$\selectlanguage{brazil}%
			\tabularnewline
			\hline 
			c & $(3,658\pm0,009)$ & \selectlanguage{english}%
			$\cdot10^{-2}\unitfrac{m}{s}$\selectlanguage{brazil}%
			\tabularnewline
			\hline 
			d & $(4,88\pm0,04)$ & \selectlanguage{english}%
			$\cdot10^{-2}\unitfrac{m}{s}$\selectlanguage{brazil}%
			\tabularnewline
			\hline 
			e & $(6,26\pm0,04)$ & \selectlanguage{english}%
			$\cdot10^{-2}\unitfrac{m}{s}$\selectlanguage{brazil}%
			\tabularnewline
			\hline 
		\end{tabular}
	\end{table}



	\subsection{Cálculo do Fator de Correção de Ladenburg para Cada Esfera}

		O fator de Ladenburg é dado pela fórmula:
	
		\[
		K=\left(1+2,4\cdot\cfrac{r}{R}\right)\cdot\left(1+3,3\cdot\cfrac{r}{h}\right)
		\]


		cujo erro associado é dado por:

		\[
		\sigma_{K}=\sqrt{\left(\sigma_{r}\cdot\cfrac{\partial K}{\partial r}\right)^{2}+\left(\sigma_{R}\cdot\cfrac{\partial K}{\partial R}\right)^{2}+\left(\sigma_{h}\cdot\cfrac{\partial K}{\partial h}\right)^{2}
		\]

		\begin{adjustwidth}{-1cm}{-1cm}
		\[
		$
		\sigma_{K}=\sqrt{\sigma_{h}^{2}\cdot10,89\cdot\cfrac{r^{2}}{h^{4}}\cdot\left(1+2,4\cdot\cfrac{r}{R}\right)^{2}+\sigma_{r}^{2}\cdot\left(\cfrac{1}{h}\left(3.3+7,92\cdot\cfrac{r}{R}\right)+\cfrac{1}{R}\left(2.4+7,92\cdot\cfrac{r}{h}\right)\right)^{2}+\sigma_{R}^{2}\cdot5,76\cdot\cfrac{r^{2}}{R^{4}}\cdot\left(1+3,3\cdot\cfrac{r}{h}\right)^{2}}
		$
		\]
		\end{adjustwidth}


		Portanto, aplicando essa fórmula pra cada esfera tem-se que:

		\begin{table}[H]
			\caption{Valores do Fator de Correção de Ladenburg, $K$, calculados à partir
			dos raios das esferas, do raio do tubo e da altura de liquido}


			\centering{}%
			\begin{tabular}{|c|c|c|}
				\hline 
				Esfera & Raio das esferas ($r$)  & Fator de Correção de Ladenburg ($K$)\tabularnewline
				\hline 
				a & $(1,0\pm0,3)\cdot10^{-3}\unit{m}$ & $1,1\pm0,2$\tabularnewline
				\hline 
				b & $(1,3\pm0,3)\cdot10^{-3}\unit{m}$ & $1,1\pm0,2$\tabularnewline
				\hline 
				c & $(1,5\pm0,3)\cdot10^{-3}\unit{m}$ & $1,2\pm0,2$\tabularnewline
				\hline 
				d & $(1,8\pm0,3)\cdot10^{-3}\unit{m}$ & $1,2\pm0,2$\tabularnewline
				\hline 
				e & $(2,0\pm0,3)\cdot10^{-3}\unit{m}$ & $1,2\pm0,2$\tabularnewline
				\hline 
			\end{tabular}
		\end{table}

	\subsection{Gráfico de $v'_{L}\times r^{2}$ e $v_{L}\times r^{2}$}

		\[
			v_{L}=K\cdot v'_{L}\overset{K\neq0}{\Longleftrightarrow}v'_{L}=\cfrac{v_{L}}{K}
		\]


		\[
		\begin{cases}
			\underset{y_{v_{L}}}{\underbrace{v{}_{L}}} & =\underset{a_{v_{L}}}{\underbrace{\cfrac{2}{9}\cdot\cfrac{\rho_{e}-\rho_{f}}{\eta}\cdot g}}\cdot\underset{x_{v_{L}}}{\underbrace{r^{2}}}+\underset{b_{v_{L}}}{\underbrace{0}}\\
			\underset{y_{v'_{L}}}{\underbrace{v'{}_{L}}} & =\underset{a_{v'_{L}}}{\underbrace{\cfrac{a_{v_{L}}}{K}}}\cdot\underset{x_{v'_{L}}}{\underbrace{r^{2}}}+\underset{b_{v'_{L}}}{\underbrace{0}}
		\end{cases}\Leftrightarrow a_{v'_{L}}=\cfrac{a_{v_{L}}}{K}\Leftrightarrow a_{v_{L}}=K\cdot a_{v'_{L}}
		\]


		assim:

		$\sigma_{a_{v_{L}}}=\sqrt{\left(\sigma_{K}\cdot\cfrac{\partial a_{v_{L}}}{\partial K}\right)^{2}+\left(\sigma_{a_{v'_{L}}}\cdot\cfrac{\partial a_{v_{L}}}{\partial a_{v'_{L}}}\right)^{2}}=\sqrt{\left(\sigma_{K}\cdot a_{v'_{L}}\right)^{2}+\left(\sigma_{a_{v'_{L}}}\cdot K\right)^{2}}$

		Para determinar $a_{v'_{L}}$usamos a fórmula:

		\selectlanguage{english}%
		\[
			\underset{y_{v'_{L}}}{\underbrace{v'{}_{L}}}=\underset{a_{v'_{L}}}{\underbrace{\cfrac{a_{v_{L}}}{K}}}\cdot\underset{x_{v'_{L}}}{\underbrace{r^{2}}}+\underset{b_{v'_{L}}}{\underbrace{0}}=\underset{a_{v_{L}}}{\underbrace{\cfrac{2}{9}\cdot\cfrac{\rho_{e}-\rho_{f}}{\eta\cdot K}\cdot g}}\cdot\underset{x_{v_{L}}}{\underbrace{r^{2}}}+\underset{b_{v_{L}}}{\underbrace{0}}
		\]


		\selectlanguage{brazil}%

	\subsection{Obtenção do Coeficiente de Viscosidade ($\eta\pm\sigma_{\eta}$)}

	\subsection{Estimativa da Concentração de Glicerina}


\section{Conclusão}

	Conclusão here

\end{document}
