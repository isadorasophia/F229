\documentclass[a4paper]{article}

% algumas packages para arrumar as tables co a margin:
% allows for temporary adjustment of side margins
\usepackage{changepage}

% provides filler text
\usepackage{lipsum}

% just makes the table prettier (see \toprule, \bottomrule, etc. commands below)
\usepackage{booktabs}

\usepackage[english, brazil]{babel} %Para traduzir os textos
\usepackage[utf8]{inputenc} %Para poder usar acentos
\usepackage[a4paper]{geometry} %Para ajustar a parte geometrica da folha
\geometry{verbose,tmargin=2cm,bmargin=3cm,lmargin=15mm,rmargin=15mm} %Parte de margens
\setlength{\parindent}{10mm}
\usepackage{wrapfig} %Biblioteca Matematica/Grafica
\usepackage{mathptmx} %Biblioteca Matematica/Grafica
\renewcommand{\ttdefault}{mathptmx} %Biblioteca Matematica/Grafica
\usepackage{amsmath} %Biblioteca Matematica/Grafica
\usepackage{amssymb} %Biblioteca Matematica/Grafica
\usepackage[11pt]{moresize}% different letters sizes
\usepackage{float}% enables accurate location of tables
\usepackage{caption}% to make personalized captions
\usepackage{graphicx} %Para inclusão de imagens
\usepackage{amsfonts}
\usepackage[T1]{fontenc}

%Packages do Lyx:
	\usepackage{units}

	%% Because html converters don't know tabularnewline
	\providecommand{\tabularnewline}{\\}
%Fim dos Packages do Lyx

\makeatletter
\providecommand{\tabularnewline}{\\} %define

\title{Calorimetria \\ Experimento 6a} % main title
\author{F 229 \\ \textsc{Grupo 1}}
\date{19 de Novembro, 2014}

\begin{document} % actually starts the document here
\maketitle

% members of the group
\begin{center}
	\begin{tabular}{rr}
		                     Integrantes: & \tabularnewline
		                                  & \tabularnewline
		Henrique Noronha Facioli          & RA: 157986 \tabularnewline
		Guilherme Lucas da Silva          & RA: 155618 \tabularnewline
		Beatriz Sechin Zazulla            & RA: 154779 \tabularnewline
		Lucas Alves Racoci                & RA: 156331 \tabularnewline
		Isadora Sophia Garcia Rodopoulos  & RA :158018 \tabularnewline
	\end{tabular}
\par\end{center}

%%Seria legal se alguém conseguir colocar os logos no footer da pagina, ou então no header...

\begin{figure}[!ht]
	\begin{minipage}[b]{0.45\linewidth}
		\includegraphics[scale=0.25]{logo-unicamp-name-line-blk-blk-0480.jpg}
	\end{minipage}
	\hspace{0.5cm}
	\begin{minipage}[b]{0.45\linewidth}
		\includegraphics[scale=0.25]{logo-ifgw.png}
	\end{minipage}
\end{figure}

\newpage{}

\section{Resumo}
	Neste experimento, estudamos o calorímetro, um instrumento usado na medição
	de processos que envolvam trocam de calor, ou seja, mudança de temperatura, estado ou 
	qualquer outro processo que seja alterado pelo calor no sistema. Dividido em duas partes, 
	a primeira do experimento (alvo desse relatório) consistiu em achar a o gráfico de 
	calibração do termopar, mergulhando um fio do termopar em um recipiente com água 
	próximo a zero grau Celsius, outro no calorímetro com água aquecida, medindo, então, 
	pares de diferença de potencial em Volts e temperatura em Celsius. Além disso, obtemos 
	a capacidade térmica do calorímetro, misturando duas quantidades de água em 
	temperaturas diferentes dentro do calorímetro e observando sua temperatura de 
	equilíbrio. Por último, construímos um gráfico linearizado para a obtenção da constante de 
	tempo do calorímetro, com os dados de temperatura em função do tempo obtidos no 
	laboratório, com o auxílio de um termômetro de mercúrio e um cronômetro, 
	respectivamente.

\section{Objetivo}
	Esse experimento tem como principal objetivo o estudo do calorímetro. Para que
	esse estudo fosse bem sucedido, determinados três aspectos do mesmo: o gráfico de 
	calibração do termopar, a partir dos valores de volts, dados pelo termopar, e de 
	temperatura, obtidos a partir do termômetro. Assim, o gráfico será plotado em função de V 
	x T. Além desse gráfico, também procuramos a constante de tempo do calorímetro. Essa 
	constante é necessária, uma vez que o instrumento que possuímos no laboratório não é 
	ideal, ou seja, a temperatura em seu interior cai conforme o tempo passa. A constante do 
	tempo mostra o quanto o calorímetro não é ideal. Por último, procuramos pela capacidade 
	térmica, a partir da temperatura de equilíbrio de uma mistura de água quente e fria.
	
\section{Procedimentos e coleta de dados}
	Para realizar o experimento necessitamos dos seguintes materiais:
	\subsection{Materiais utilizados}
		\begin{enumerate}
			\item Calorímetro
			\item Termômetro de mercurio
			\item Cronômetro
			\item Termopar
			\item Milivoltimetro
		\end{enumerate}

	\subsection{Procedimento}
	De posse dos materiais, primeiramente fizemos a calibração do termopar, e para isso colocamos em um béquer água com gelo e dentro do calorímetro água fervente e, para alguns valores da temperatura $T$ que ia variando, verificavamos $\Delta V$ com o termopar, obtendo a seguinte tabela:
	
\begin{table}[!ht]
	\caption{Valores obtidos de diferença de potencial $V$ elétrico em função da temperatura $T$}
	\begin{center}\begin{tabular}{|c|c|}
	\hline 
	Temperatura ($T$)  & Diferença de Potencial ($\Delta V$)\tabularnewline
	\hline 
	$(8,50\pm0,05)\cdot10^{1}\unit{^{\circ}C}$  & $(4,84\pm0,01)\cdot10^{-3}\unit{V}$\tabularnewline
	\hline 
	$(8,00\pm0,05)\cdot10^{1}\unit{^{\circ}C}$  & $(4,45\pm0,01)\cdot10^{-3}\unit{V}$\tabularnewline
	\hline 
	$(7,50\pm0,05)\cdot10^{1}\unit{^{\circ}C}$  & $(4,02\pm0,01)\cdot10^{-3}\unit{V}$\tabularnewline
	\hline 
	$(7,00\pm0,05)\cdot10^{1}\unit{^{\circ}C}$  & $(3,82\pm0,01)\cdot10^{-3}\unit{V}$\tabularnewline
	\hline 
	$(6,50\pm0,05)\cdot10^{1}\unit{^{\circ}C}$  & $(3,52\pm0,01)\cdot10^{-3}\unit{V}$\tabularnewline
	\hline 
	$(6,00\pm0,05)\cdot10^{1}\unit{^{\circ}C}$  & $(3,23\pm0,01)\cdot10^{-3}\unit{V}$\tabularnewline
	\hline 
	$(5,50\pm0,05)\cdot10^{1}\unit{^{\circ}C}$  & $(2,95\pm0,01)\cdot10^{-3}\unit{V}$\tabularnewline
	\hline 
\end{tabular}\end{center}
\end{table}

\begin{table}[!ht]
	\caption{Valor da temperatura ($T_{gelo}$) do béquer contendo água e gelo}
	\begin{center}
	\begin{tabular}{|c|c|}
	\hline
	Temperatura da agua + gelo & $(2\pm5)\cdot10^{-1}\unit{^{\circ}C}$\tabularnewline
	\hline 
	\end{tabular}
	\end{center}
\end{table}
	
	Após a calibração do termopar, precisávamos encontrar a constante do tempo $\tau$, e para isso colocamos agua fervente no interior do calorímetro e verificamos a temperatura $T$ pelo tempo $t$, obtendo a seguinte tabela:
	
\begin{table}[!ht]
	\begin{center}
	\caption{Valor da temperatura ($T$) do calorímetro pelo tempo $t$}
    \begin{tabular}{|l|l|}
    \hline
    Temperatura(C) & Tempo (min) \\ \hline
    90             & 0           \\ \hline
    87,5           & 1min 5s     \\ \hline
    85             & 3min 58s    \\ \hline
    82,5           & 6 min 2s    \\ \hline
    80             & 9min 35s    \\ \hline
    75             & 16min 48s   \\ \hline
    70             & 26min 50s   \\ \hline
    \end{tabular}
    \end{center}
\end{table}

\section{Análise dos resultados}	
	\subsection{Gráfico de Calibração do Termopar}
		\subsubsection{Achando a relação entre valores de$\Delta V$ e valores de $\Delta T$}
		À partir da Tabela 1 e da temperatura $T_{gelo}$ do bequer agua+gelo, tem-se que:

\begin{table}[H]
	\caption{Calculo de $\Delta T$}
	\begin{centering}\begin{tabular}{|c|}
		\hline 
		Diferença de Temperatura ($\Delta T$) \tabularnewline
		\hline 
		$\left((8,50\pm0,05)-(0,02\pm0,05)\right)\cdot10^{1}\unit{^{\circ}C}$\tabularnewline
		\hline 
		$\left((8,00\pm0,05)-(0,02\pm0,05)\right)\cdot10^{1}\unit{^{\circ}C}$\tabularnewline
		\hline 
		$\left((7,50\pm0,05)-(0,02\pm0,05)\right)\cdot10^{1}\unit{^{\circ}C}$\tabularnewline
		\hline 
		$\left((7,00\pm0,05)-(0,02\pm0,05)\right)\cdot10^{1}\unit{^{\circ}C}$\tabularnewline
		\hline 
		$\left((6,50\pm0,05)-(0,02\pm0,05)\right)\cdot10^{1}\unit{^{\circ}C}$\tabularnewline
		\hline 
		$\left((6,00\pm0,05)-(0,02\pm0,05)\right)\cdot10^{1}\unit{^{\circ}C}$\tabularnewline
		\hline 
		$\left((5,50\pm0,05)-(0,02\pm0,05)\right)\cdot10^{1}\unit{^{\circ}C}$\tabularnewline
		\hline 
	\end{tabular}
				$\Longleftrightarrow$
	\begin{tabular}{|c|}
		\hline 
		Diferença de Temperatura ($\Delta T$) \tabularnewline
		\hline 
		$(8,48\pm0,05\sqrt{2})\cdot10^{1}\unit{K}$\tabularnewline
		\hline 
		$(7,98\pm0,05\sqrt{2})\cdot10^{1}\unit{K}$\tabularnewline
		\hline 
		$(7,48\pm0,05\sqrt{2})\cdot10^{1}\unit{K}$\tabularnewline
		\hline 
		$(6,98\pm0,05\sqrt{2})\cdot10^{1}\unit{K}$\tabularnewline
		\hline 
		$(6,48\pm0,05\sqrt{2})\cdot10^{1}\unit{K}$\tabularnewline
		\hline 
		$(5,98\pm0,05\sqrt{2})\cdot10^{1}\unit{K}$\tabularnewline
		\hline 
		$(5,48\pm0,05\sqrt{2})\cdot10^{1}\unit{K}$\tabularnewline
		\hline 
	\end{tabular}\par\end{centering}
				\[
				\Updownarrow
				\]
	\centering{}\begin{tabular}{|c|c|}
		\hline 
		Diferença de Temperatura ($\Delta T$)  & Diferença de Potencial ($\Delta V$)\tabularnewline
		\hline 
		$(8,48\pm0,07)\cdot10^{1}\unit{K}$  & $(4,84\pm0,01)\cdot10^{-3}\unit{V}$\tabularnewline
		\hline 
		$(7,98\pm0,07)\cdot10^{1}\unit{K}$  & $(4,45\pm0,01)\cdot10^{-3}\unit{V}$\tabularnewline
		\hline 
		$(7,48\pm0,07)\cdot10^{1}\unit{K}$  & $(4,02\pm0,01)\cdot10^{-3}\unit{V}$\tabularnewline
		\hline 
		$(6,98\pm0,07)\cdot10^{1}\unit{K}$  & $(3,82\pm0,01)\cdot10^{-3}\unit{V}$\tabularnewline
		\hline 
		$(6,48\pm0,07)\cdot10^{1}\unit{K}$  & $(3,52\pm0,01)\cdot10^{-3}\unit{V}$\tabularnewline
		\hline 
		$(5,98\pm0,07)\cdot10^{1}\unit{K}$  & $(3,23\pm0,01)\cdot10^{-3}\unit{V}$\tabularnewline
		\hline 
		$(5,48\pm0,07)\cdot10^{1}\unit{K}$  & $(2,95\pm0,01)\cdot10^{-3}\unit{V}$\tabularnewline
		\hline 
	\end{tabular}
\end{table}
		\subsubsection{Método dos Mínimos Quadrados}

			Para realizar o método dos mínimos quadrados faremos $y=\Delta V$ e $x=\Delta T$

			Sabe-se que $\underset{y}{\underbrace{\Delta V}}=\underset{a}{\underbrace{\left(S_{a}(T)-S_{B}(T)\right)}}\underset{x}{\underbrace{\Delta T}}+\underset{b}{\underbrace{0}}$.

			Como nesse caso o erro em $y=\Delta V$ é constante, o método dos
			mínimos quadrados usado será o que é dado pelas fórmulas:

			\[
			\begin{cases}
				a & =\frac{\left({\displaystyle {\displaystyle N\cdot}{\scriptscriptstyle {\textstyle \overset{n}{\underset{i=1}{\sum}}}}}\left(x_{i}y_{i}\right)-{\displaystyle {\scriptscriptstyle {\textstyle \overset{n}{\underset{i=1}{\sum}}}}}\left(x_{i}\right){\scriptscriptstyle {\textstyle \overset{n}{\cdot\underset{i=1}{\sum}}}}\left(y_{i}\right)\right)}{{\displaystyle {\displaystyle N\cdot}{\scriptscriptstyle {\textstyle \overset{n}{\underset{i=1}{\sum}}}}}\left(x_{i}^{2}\right)-\left({\displaystyle {\scriptscriptstyle {\textstyle \overset{n}{\underset{i=1}{\sum}}}}}\left(x_{i}\right)\right)^{2}}\pm\sigma_{y}\sqrt{\cfrac{{\displaystyle {\displaystyle {\displaystyle N}}}}{{\displaystyle {\displaystyle N\cdot}{\scriptscriptstyle {\textstyle \overset{n}{\underset{i=1}{\sum}}}}}\left(x_{i}^{2}\right)-\left({\displaystyle {\scriptscriptstyle {\textstyle \overset{n}{\underset{i=1}{\sum}}}}}\left(x_{i}\right)\right)^{2}}}\\
				b & =\frac{\left({\displaystyle {\scriptscriptstyle {\textstyle \overset{n}{\underset{i=1}{\sum}}}}}\left(y_{i}\right)\cdot{\displaystyle {\scriptscriptstyle {\textstyle \overset{n}{\underset{i=1}{\sum}}}}}\left(x_{i}^{2}\right)-{\displaystyle {\scriptscriptstyle {\textstyle \overset{n}{\underset{i=1}{\sum}}}}}\left(x_{i}y_{i}\right){\scriptscriptstyle {\textstyle \overset{n}{\cdot\underset{i=1}{\sum}}}}\left(x_{i}\right)\right)}{N\cdot{\displaystyle {\scriptscriptstyle {\textstyle \overset{n}{\underset{i=1}{\sum}}}}}\left(x_{i}^{2}\right)-\left({\displaystyle {\scriptscriptstyle {\textstyle \overset{n}{\underset{i=1}{\sum}}}}}\left(x_{i}\right)\right)^{2}}\pm\sigma_{y}\sqrt{\cfrac{{\displaystyle {\scriptscriptstyle {\textstyle \overset{n}{\underset{i=1}{\sum}}}}}\left(x_{i}^{2}\right)}{{\displaystyle {\displaystyle N\cdot}{\scriptscriptstyle {\textstyle \overset{n}{\underset{i=1}{\sum}}}}}\left(x_{i}^{2}\right)-\left({\displaystyle {\scriptscriptstyle {\textstyle \overset{n}{\underset{i=1}{\sum}}}}}\left(x_{i}\right)\right)^{2}}}
			\end{cases}
			\]


			Que pode ser calculado mais facilmente atravez de uma tabela auxiliar
			como a seguinte:
			\begin{table}
				\caption{Tabela auxiliar para o cálculo dos mínimos quadrados:}

				\begin{center}\begin{tabular}{|c|c|c|c|c|}
					\hline 
					N & $y=\Delta V$ & $x=\Delta T$ & $xy$ & $x^{2}$\tabularnewline
					\hline 
					$1$ & \selectlanguage{english}%
					$4,84\cdot10^{-3}\unit{V}$\selectlanguage{brazil}%
					 & $8,48\cdot10^{1}\unit{K}$ & $4,10\cdot10^{-1}\unit{K\cdot V}$ & $7,19\cdot10^{3}\unit{K^{2}}$\tabularnewline
					\hline 
					$2$ & \selectlanguage{english}%
					$4,45\cdot10^{-3}\unit{V}$\selectlanguage{brazil}%
					 & $7,98\cdot10^{1}\unit{K}$ & $3,55\cdot10^{-1}\unit{K\cdot V}$ & $6,37\cdot10^{3}\unit{K^{2}}$\tabularnewline
					\hline 
					$3$ & \selectlanguage{english}%
					$4,02\cdot10^{-3}\unit{V}$\selectlanguage{brazil}%
					 & $7,48\cdot10^{1}\unit{K}$ & $3,01\cdot10^{-1}\unit{K\cdot V}$ & $5,60\cdot10^{3}\unit{K^{2}}$\tabularnewline
					\hline 
					$4$ & \selectlanguage{english}%
					$3,82\cdot10^{-3}\unit{V}$\selectlanguage{brazil}%
					 & $6,98\cdot10^{1}\unit{K}$ & $2,67\cdot10^{-1}\unit{K\cdot V}$ & $4,87\cdot10^{3}\unit{K^{2}}$\tabularnewline
					\hline 
					$5$ & \selectlanguage{english}%
					$3,52\cdot10^{-3}\unit{V}$\selectlanguage{brazil}%
					 & $6,48\cdot10^{1}\unit{K}$ & $2,28\cdot10^{-1}\unit{K\cdot V}$ & $4,20\cdot10^{3}\unit{K^{2}}$\tabularnewline
					\hline 
					$6$ & \selectlanguage{english}%
					$3,23\cdot10^{-3}\unit{V}$\selectlanguage{brazil}%
					 & $5,98\cdot10^{1}\unit{K}$ & $1,93\cdot10^{-1}\unit{K\cdot V}$ & $3,58\cdot10^{3}\unit{K^{2}}$\tabularnewline
					\hline 
					$7$ & \selectlanguage{english}%
					$2,95\cdot10^{-3}\unit{V}$\selectlanguage{brazil}%
					 & $5,48\cdot10^{1}\unit{K}$ & $1,62\cdot10^{-1}\unit{K\cdot V}$ & $3,00\cdot10^{3}\unit{K^{2}}$\tabularnewline
					\hline 
					$\Sigma$ & \selectlanguage{english}%
					$2,68\cdot10^{-2}\unit{V}$\selectlanguage{brazil}%
					 & $4,89\cdot10^{2}\unit{K}$ & $1,92\unit{K\cdot V}$ & $3,48\cdot10^{4}\unit{K^{2}}$\tabularnewline
					\hline 
				\end{tabular}\par\end{center}

			\end{table}


			Assim obtemos:

			\[
				\begin{cases}
					a & =\left(6,15\pm0,04\right)\cdot10^{-5}\unit{\frac{V}{K}}\\
					b & =\left(0\pm0\right)\cdot\unit{V}
				\end{cases}
			\]

		\subsubsection{Construção do Gráfico}

			\begin{figure}
				\caption{Gráfico da curva de calibração do termopar de $\Delta V$ por $\Delta T$}
					\begin{centering}
						\includegraphics[scale=0.5]{image}
					\par\end{centering}
			\end{figure}

	\subsection{Constante de Tempo do Calorímetro}

	\subsection{Capacidade Térmica do Calorímetro}


		\subsubsection{Equação Literal para Obtenção da Capacidade Térmica ($C$)}

			Começemos com a fórmula:

			\[
				Q_{resultante}={\displaystyle \sum Q_{parciais}=0}\Leftrightarrow
			\]


			\[
				Q_{calor\acute{\imath}metro}+Q_{\acute{a}gua_{fria}}+Q_{\acute{a}gua_{quente}}=0
			\]


			Sendo $T_{f}$ a temperatura da água fria, $T_{q}$ a temperatura
			da água quente, $T_{e}$a temperatura de equilibrio termodinâmico
			e $C$ é a capacidade térmica do calorímetro:

			\[
			\begin{cases}
				Q_{calor\acute{\imath}metro} & =C\centerdot(T_{e}-T_{f})\\
				Q_{\acute{a}gua_{fria}} & =M_{f}c_{\acute{a}gua}\centerdot(T_{e}-T_{f})\\
				Q_{\acute{a}gua_{quente}} & =M_{q}c_{\acute{a}gua}\centerdot(T_{e}-T_{q})
			\end{cases}
			\]


			Portanto substituindo isso na equação anterior:

			$C\centerdot(T_{e}-T_{f})+M_{f}\centerdot c_{\acute{a}gua}\centerdot(T_{e}-T_{f})+M_{q}\centerdot c_{\acute{a}gua}\centerdot(T_{e}-T_{q})=0\Longleftrightarrow$

			$C\centerdot(T_{e}-T_{f})=M_{q}\centerdot c_{\acute{a}gua}\centerdot(T_{q}-T_{e})-M_{f}\centerdot c_{\acute{a}gua}\centerdot(T_{e}-T_{f})\Longleftrightarrow$

			$C=c_{\acute{a}gua}\centerdot\dfrac{M_{q}\centerdot(T_{q}-T_{e})-M_{f}\centerdot(T_{e}-T_{f})}{T_{e}-T_{f}}=c_{\acute{a}gua}\centerdot\left(M_{q}\centerdot\dfrac{T_{q}-T_{e}}{T_{e}-T_{f}}-M_{f}\right)=c_{\acute{a}gua}\centerdot\left(M_{q}\centerdot\dfrac{\Delta T_{e\rightarrow q}}{\Delta T_{f\rightarrow e}}-M_{f}\right)$

			Para calcular o erro na capacidade térmica do calorímetro $\sigma_{C}$
			teríamos que usar a fórmula:

			$\sigma_{C}^{2}=\left(\cfrac{\partial C}{\partial M_{f}}\right)^{2}\sigma_{M_{f}}^{2}+\left(\cfrac{\partial C}{\partial M_{q}}\right)^{2}\sigma_{M_{q}}^{2}+\left(\cfrac{\partial C}{\partial(\Delta T_{e\rightarrow q})}\right)^{2}\sigma_{\Delta T_{e\rightarrow q}}^{2}+\left(\cfrac{\partial C}{\partial(\Delta T_{f\rightarrow e})}\right)^{2}\sigma_{\Delta T_{f\rightarrow e}}^{2}$

			Mas sabemos que $\sigma_{\Delta T_{e\rightarrow q}}^{2}=\sigma_{\Delta T_{f\rightarrow e}}^{2}=\sigma_{\Delta T}^{2}=\left(\cfrac{\partial(T_{2}-T_{1})}{\partial T_{1}}\right)^{2}\sigma_{T}^{2}+\left(\cfrac{\partial(T_{2}-T_{1})}{\partial T_{2}}\right)^{2}\sigma_{T}^{2}=\left(\left(-1\right)^{2}+1^{2}\right)\sigma_{T}^{2}=2\centerdot\sigma_{T}^{2}$.

			Também sabemos que $\sigma_{M_{q}}=\sigma_{M_{f}}=\sigma_{M}$.

			Então:

			\[
				\sigma_{C}=\sqrt{\left(\left(-c_{\acute{a}gua}\right)^{2}+\left(c_{\acute{a}gua}\centerdot\dfrac{\Delta T_{e\rightarrow q}}{\Delta T_{f\rightarrow e}}\right)^{2}\right)\sigma_{M}^{2}+2\centerdot\left(\left(c_{\acute{a}gua}\centerdot M_{q}\centerdot\cfrac{1}{\Delta T_{f\rightarrow e}}\right)^{2}+\left(-c_{\acute{a}gua}\centerdot M_{q}\centerdot\cfrac{\Delta T_{e\rightarrow q}}{\Delta T_{f\rightarrow e}^{2}}\right)^{2}\right)\sigma_{T}^{2}}
			\]


			\[
				=c_{\acute{a}gua}\centerdot\sqrt{\left(1+\left(\dfrac{\Delta T_{e\rightarrow q}}{\Delta T_{f\rightarrow e}}\right)^{2}\right)\sigma_{M}^{2}+2\centerdot M_{q}^{2}\centerdot\left(\left(\cfrac{1}{\Delta T_{f\rightarrow e}}\right)^{2}+\left(\cfrac{\Delta T_{e\rightarrow q}}{\Delta T_{f\rightarrow e}^{2}}\right)^{2}\right)\sigma_{T}^{2}}
			\]


			\[
				=c_{\acute{a}gua}\centerdot\sqrt{\left(\dfrac{\Delta T_{f\rightarrow e}^{2}+\Delta T_{e\rightarrow q}^{2}}{\Delta T_{f\rightarrow e}^{2}}\right)\sigma_{M}^{2}+2\centerdot M_{q}^{2}\centerdot\left(\cfrac{\Delta T_{f\rightarrow e}^{2}+\Delta T_{e\rightarrow q}^{2}}{\Delta T_{f\rightarrow e}^{4}}\right)\sigma_{T}^{2}}
			\]


			\[
				=c_{\acute{a}gua}\centerdot\sqrt{\left(\dfrac{\Delta T_{f\rightarrow e}^{2}+\Delta T_{e\rightarrow q}^{2}}{\Delta T_{f\rightarrow e}^{4}}\right)\left(\Delta T_{f\rightarrow e}^{2}\centerdot\sigma_{M}^{2}+2\centerdot M_{q}^{2}\centerdot\sigma_{T}^{2}\right)}
			\]


			\[
				=\cfrac{c_{\acute{a}gua}}{\Delta T_{f\rightarrow e}^{2}}\centerdot\sqrt{\Delta T_{f\rightarrow e}^{2}+\Delta T_{e\rightarrow q}^{2}}\sqrt{\Delta T_{f\rightarrow e}^{2}\centerdot\sigma_{M}^{2}+2\centerdot M_{q}^{2}\centerdot\sigma_{T}^{2}}
			\]


			\[
				=c_{\acute{a}gua}\centerdot\cfrac{\sqrt{\Delta T_{f\rightarrow e}^{2}+\Delta T_{e\rightarrow q}^{2}}\sqrt{\left(\Delta T_{f\rightarrow e}\centerdot\sigma_{M}\right)^{2}+2\centerdot\left(M_{q}\centerdot\sigma_{T}\right)^{2}}}{\Delta T_{f\rightarrow e}^{2}}
			\]


			Portanto:

			\[
			C=c_{\acute{a}gua}\centerdot\left(\left(M_{q}\centerdot\dfrac{\Delta T_{e\rightarrow q}}{\Delta T_{f\rightarrow e}}-M_{f}\right)\pm\left(\cfrac{\sqrt{\Delta T_{f\rightarrow e}^{2}+\Delta T_{e\rightarrow q}^{2}}\sqrt{\left(\Delta T_{f\rightarrow e}\centerdot\sigma_{M}\right)^{2}+2\centerdot\left(M_{q}\centerdot\sigma_{T}\right)^{2}}}{\Delta T_{f\rightarrow e}^{2}}\right)\right)
			\]

		\subsubsection{Resultado das Medidas}

			\begin{table}
				\caption{Valores brutos obtidos e conversão para as únidades a serem ultilizads}


				\begin{center}\begin{tabular}{|c|c|r|r|}
					\hline 
					Nomes dos Dados Obtidos  & Símbolo Associado  & Valor Obtido  & Valor Ultilizado\tabularnewline
					\hline 
					\emph{Massa do Béquer Vazio}  & $M_{b\acute{e}quer}$  & $(1,033\pm0,001)\centerdot10^{2}\unit{g}$  & $(1,033\pm0,001)\centerdot10^{-1}\unit{Kg}$\tabularnewline
					\hline 
					\emph{Massa do Béquer com Água Fria}  & $M_{b\acute{e}quer}+M_{f}$  & \selectlanguage{english}%
					$(3,021\pm0,001)\centerdot10^{2}\unit{g}$\selectlanguage{brazil}%
					 & \selectlanguage{english}%
					$(3,021\pm0,001)\centerdot10^{-1}\unit{Kg}$\selectlanguage{brazil}%
					\tabularnewline
					\hline 
					\emph{Massa do Béquer com Água Quente}  & $M_{b\acute{e}quer}+M_{q}$  & \selectlanguage{english}%
					$(2,721\pm0,001)\centerdot10^{2}\unit{g}$\selectlanguage{brazil}%
					 & \selectlanguage{english}%
					$(2,721\pm0,001)\centerdot10^{-1}\unit{Kg}$\selectlanguage{brazil}%
					\tabularnewline
					\hline 
					Temperatura da Água Fria  & $T_{f}$  & \selectlanguage{english}%
					$(3,10\pm0,05)\centerdot10^{1}\unit{^{\circ}C}$\selectlanguage{brazil}%
					 & \selectlanguage{english}%
					$(3,10\pm0,05)\centerdot10^{1}\unit{^{\circ}C}$\selectlanguage{brazil}%
					\tabularnewline
					\hline 
					Temperatura da Água em Equilíbrio  & $T_{e}$  & \selectlanguage{english}%
					$(5,50\pm0,05)\centerdot10^{1}\unit{^{\circ}C}$\selectlanguage{brazil}%
					 & \selectlanguage{english}%
					$(5,50\pm0,05)\centerdot10^{1}\unit{^{\circ}C}$\selectlanguage{brazil}%
					\tabularnewline
					\hline 
					Temperatura da Água Quente  & $T_{q}$  & \selectlanguage{english}%
					$(8,50\pm0,05)\centerdot10^{1}\unit{^{\circ}C}$\selectlanguage{brazil}%
					 & \selectlanguage{english}%
					$(8,50\pm0,05)\centerdot10^{1}\unit{^{\circ}C}$\selectlanguage{brazil}%
					\tabularnewline
					\hline 
				\end{tabular}\end{center}
			\end{table}


			Temos portanto:

			\[
			\begin{cases}
				\begin{cases}
					M_{f} & =\left[M_{b\acute{e}quer}+M_{f}\right]-\left[M_{b\acute{e}quer}\right]\\
					M_{q} & =\left[M_{b\acute{e}quer}+M_{q}\right]-\left[M_{b\acute{e}quer}\right]
				\end{cases} & \underleftrightarrow{matricialmente}\\
				\begin{cases}
					\Delta T_{e\rightarrow q} & =T_{q}-T_{e}\\
					\Delta T_{f\rightarrow e} & =T_{e}-T_{f}
				\end{cases} & \underleftrightarrow{matricialmente}
			\end{cases}\Longleftrightarrow
			\]


			\[
			\begin{cases}
			\left[\begin{array}{c}
				M_{f}\\
				M_{q}
			\end{array}\right] & =\left(\left[\begin{array}{c}
					\left(3,021\pm0,001\right)\\
					\left(2,721\pm0,001\right)
				\end{array}\right]-\left[\begin{array}{c}
					\left(1,033\pm0,001\right)\\
					\left(1,033\pm0,001\right)
			\end{array}\right]\right)\cdot\left[\begin{array}{c}
				10^{-1}\unit{Kg}\end{array}\right]\\
			\left[\begin{array}{c}
				\Delta T_{e\rightarrow q}\\
				\Delta T_{f\rightarrow e}
			\end{array}\right] & =\left[\begin{array}{c}
				T_{q}\\
				T_{e}
			\end{array}\right]\cdot\left[\begin{array}{c}
				\unit{^{\circ}C}\end{array}\right]-\left[\begin{array}{c}
				T_{e}\\
				T_{f}
			\end{array}\right]\cdot\left[\begin{array}{c}
				\unit{^{\circ}C}\end{array}\right]
			\end{cases}\Longleftrightarrow
			\]


			\[
			\begin{cases}
				\left[\begin{array}{c}
					M_{f}\\
					M_{q}
				\end{array}\right] & =\left[\begin{array}{c}
					1,988\pm0,001\cdot\sqrt{2}\\
					1,688\pm0,001\cdot\sqrt{2}
				\end{array}\right]\cdot\left[\begin{array}{c}
					10^{-1}\unit{Kg}\end{array}\right]\\
				\left[\begin{array}{c}
					\Delta T_{e\rightarrow q}\\
					\Delta T_{f\rightarrow e}
				\end{array}\right] & =\left(\left[\begin{array}{c}
						T_{q}\\
						T_{e}
					\end{array}\right]-\left[\begin{array}{c}
						T_{e}\\
						T_{f}
				\end{array}\right]\right)\cdot\left[\begin{array}{c}
					K
				\end{array}\right]
			\end{cases}\Longleftrightarrow
			\]


			\[
			\begin{cases}
				\left[\begin{array}{c}
					M_{f}\\
					M_{q}
				\end{array}\right] & =\left[\begin{array}{c}
					1,988\pm0,001\\
					1,688\pm0,001
				\end{array}\right]\cdot\left[10^{-1}\unit{Kg}\right]\\
				\left[\begin{array}{c}
					\Delta T_{e\rightarrow q}\\
					\Delta T_{f\rightarrow e}
				\end{array}\right] & =\left(\left[\begin{array}{c}
						8,50\pm0,05\\
						5,50\pm0,05
					\end{array}\right]-\left[\begin{array}{c}
						5,50\pm0,05\\
						3,10\pm0,05
				\end{array}\right]\right)\cdot\left[10^{1}\begin{array}{c}
				\begin{array}{c}
				\unit{K}\end{array}\end{array}\right]
			\end{cases}\Longleftrightarrow
			\]


			\[
			\begin{cases}
				\left[\begin{array}{c}
					M_{f}\\
					M_{q}
				\end{array}\right] & =\left[\begin{array}{c}
					1,988\pm0,001\\
					1,688\pm0,001
				\end{array}\right]\cdot\left[10^{-1}Kg\right]\\
				\left[\begin{array}{c}
					\Delta T_{e\rightarrow q}\\
					\Delta T_{f\rightarrow e}
				\end{array}\right] & =\left(\left[\begin{array}{c}
					3,00\pm0,05\\
					2,40\pm0,05
				\end{array}\right]\right)\cdot\left[\begin{array}{c}
					10^{1}\begin{array}{c}
					K\end{array}\end{array}\right]
			\end{cases}\Longleftrightarrow\begin{cases}
				M_{f}=\left(1,988\pm0,001\right)\cdot10^{-1}\unit{Kg}\\
				M_{q}=\left(1,688\pm0,001\right)\cdot10^{-1}\unit{Kg}\\
				\Delta T_{e\rightarrow q}=\left(3,00\pm0,05\right)\cdot10^{1}\begin{array}{c}
				\unit{K}\end{array}\\
				\Delta T_{f\rightarrow e}=\left(2,40\pm0,05\right)\cdot10^{1}\begin{array}{c}
				\unit{K}\end{array}
			\end{cases}
			\]

		\subsubsection{Substituição das Medidas Obtidas na Fórmula Literal}

			Substituindo os valores de $M_{f}$, $M_{q}$, $\Delta T_{e\rightarrow q}$,
			$\Delta T_{f\rightarrow e}$, e usando $c_{\acute{a}gua}=4,184\cdot10^{3}\unit{\frac{J}{Kg\cdot K}}$
			temos:

			$C=\left(5\pm3\right)\cdot10\unit{\frac{J}{K}}$ou $1,8\pm0,8\unit{\frac{cal}{K}}$

\section{Conclusão}
	Conclusão here 

\end{document}
